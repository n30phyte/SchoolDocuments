\documentclass{article}

\usepackage[letterpaper, margin=1.3cm]{geometry}
\usepackage[utf8]{inputenc}
\usepackage{siunitx}
\usepackage{mathtools}
\usepackage{multicol}

\title{ECE 203 Problem Set 6}
\author{Michael Kwok}
\begin{document}

\maketitle
\begin{multicols}{2}
\section*{1}
\subsection*{a}
\begin{align}
    \left(I_1 + 10^{-5} V_2\right) 1000 &= V_1\\
    10000 I_2 - 100 V_1 &= V_2
\end{align}

Expanding equation (1) then substituting in equation (2):

\begin{align*}
    &1000 I_1 + 0.01 V_2 = V_1\\
    &1000 I_1 + 0.01 \left(10000 I_2 - 100 V_1\right) = V_1\\
    &1000 I_1 + 100 I_2 = 2 V_1
\end{align*}

Forming the following equations:

\begin{align}
    V_1 &= 500 I_1 + 50 I_2\\
    V_2 &= -50000 I_1 + 5000 I_2
\end{align}

$$
  [Z] = 
  \begin{bmatrix}
  500 & 50\\
  -50000 & 5000
  \end{bmatrix}
  \si{\ohm}
$$
\subsection*{b}
The network is not reciprocal. $Z_{12} \neq Z_{21}$
\subsection*{c}
$$
\frac{V_1}{I_1} = R_{in}
$$
\begin{align*}
    V_2 &= -I_2 R_L\\
    &= -4000 I_2
\end{align*}
\begin{align*}
    -4000 I_2 &= -50000 I_1 + 5000 I_2\\
    -9000 I_2 &= -50000 I_1\\
    I_2 &= \frac{50}{9}I_1\\
    V_1 &= 500 I_1 + \frac{2500}{9} I_1\\
    \frac{V_1}{I_1} &= \SI{778}{\ohm}\\
    R_in &= \SI{778}{\ohm}
\end{align*}
\section*{2}
\begin{align}
    I_1 &= \frac{V_1}{100} + \frac{V_a}{500} + \frac{V_1}{250}\\
    I_2 &= \frac{V_2}{200} - \frac{V_1}{250}\\
    V_a &= 250 I_2 + V_1
\end{align}

Equation (7) into (5)

\begin{equation}
    I_1 = \frac{V_1}{100} + \frac{250 I_2 + V_1}{500} + \frac{V_1}{250}
\end{equation}
\begin{align*}
    500 I_1 &- 250 I_2 = 5 V_1 + V_1 + 2 V_1\\
    500 I_1 &- 250 I_2 = 8 V_1 \\
    500 I_1 &- 250 \left( \frac{V_2}{200} - \frac{V_1}{250} \right) = 8 V_1\\
    500 I_1 &- \frac{250}{200} V_2 + V_1 = 8 V_1\\
    7V_1 &= 500 I_1 - \frac{5}{4}V_2\\
    V_1 &= \frac{500}{7} I_1 - \frac{5}{28} V_2
\end{align*}

\begin{align*}
    I_2 &= \frac{V_2}{200} - \frac{1}{250} \frac{500}{7} I_1 + \frac{5}{28}\frac{1}{250} V_2\\
    &= -\frac{2}{7} I_1 + \frac{1}{175} V_2
\end{align*}

$$
  [h] = 
  \begin{bmatrix}
  \frac{500}{7} & -\frac{5}{28}\\
  -\frac{2}{7} & \frac{1}{175}
  \end{bmatrix}
$$

$h_{11}$ has unit of $\si{\ohm}$, $h_{22}$ has unit of $\si{\per\ohm}$

\section*{3}
$R_{eq}$ can be calculated with $V_s = 0$

$$
    V_1 = -4 I_1
$$

\begin{align}
    -4 I_1 &= 2 I_1 + V_2\\
    I_2 &= 5 I_1 + 2 V_2
\end{align}
\begin{align*}
    -6 I_1 &= V_2\\
    I_1 &= -\frac{1}{6} V_2
\end{align*}
\begin{align*}
    I_2 &= -\frac{5}{6} V_2 + 2 V_2\\
    \frac{I_2}{V_2} &= \frac{7}{6}\\
    R_{eq} &= \frac{6}{7}\si{\ohm}
\end{align*}

\section*{4}
For $y_1$, the CL circuit.
\begin{align}
I_1 = \frac{V_1}{\frac{1}{sC}} + \frac{V_1 - V_2}{sL}\\
I_2 = \frac{V_2 - V_1}{sL} + \frac{V_2}{\frac{1}{sC}}
\end{align}
\begin{align*}
    I_1 = sV_1 + \frac{1}{s} V_1 - \frac{1}{s}V_2\\
    I_2 = sV_2 + \frac{1}{s} V_2 - \frac{1}{s}V_1
\end{align*}
$$
  [y_1] = 
  \begin{bmatrix}
  \frac{s^2 +1}{s} & -\frac{1}{s}\\
  -\frac{1}{s} & \frac{s^2 +1}{s}
  \end{bmatrix}
  \si{\per\ohm}
$$
For $y_2$, the RC circuit.
\begin{align}
I_1 = \frac{V_1-V_2}{\frac{1}{sC}} + \frac{V_1}{1}\\
I_2 = \frac{V_2-V_1}{\frac{1}{sC}} + \frac{V_2}{1}
\end{align}
\begin{align*}
    I_1 = (s+1)V_1 - s V_2\\
    I_2 = -sV_1 + (s+1)V_2
\end{align*}
$$
  [y_2] = 
  \begin{bmatrix}
  s+1 & -s\\
  -s & s+1
  \end{bmatrix}
  \si{\per\ohm}
$$
$$
[y] = [y_1] + [y_2] = 
  \begin{bmatrix}
  \frac{2s^2 + s + 1}{s} & -\frac{s^2+1}{s}\\
  -\frac{s^2+1}{s} & \frac{2s^2 + s + 1}{s}
  \end{bmatrix}
  \si{\per\ohm}
$$
\section*{5}
\subsection*{a}
\begin{align}
    I_1 &= \frac{V_1}{4} + \frac{V_1 - V_2}{2}\\
    I_2 &= \frac{V_2}{3} + \frac{V_2 - V_1}{2}
\end{align}
\begin{align}
    4 I_1 &= V_1 + 2 V_1 - 2 V_2\\
    6 I_2 &= 2 V_2 + 3 V_2 - 3 V_1
\end{align}
Using Equation (18):
\begin{align*}
    6 I_2 - 5 V_2 &= - 3 V_1\\
    V_1 &= \frac{5}{3} V_2 - 2 I_2
\end{align*}

Using Equation (17):
\begin{align*}
    4 I_1 &= 5 V_2 - 6 I_2 -2 V_2\\
    4 I_1 &= 3 V_2 - 6 I_2\\
    I_1 &= \frac{3}{4} V_2 - \frac{3}{2} I_2
\end{align*}

\begin{align*}
3 V_1 &= 4 I_1 + 2 V_2\\
 V_1 &= \frac{4}{3} \left( \frac{3}{4} V_2 - \frac{3}{2} I_2\right) +\frac{2}{3} V_2\\
 &= V_2 - 2 I_2 + \frac{2}{3} V_2\\
 &= \frac{5}{3} V_2 - 2I_2
\end{align*}
$$
[t] = 
  \begin{bmatrix}
  \frac{5}{3} & 2\\
  \frac{3}{4} & \frac{3}{2}
  \end{bmatrix}
$$
$t_{12}$ has unit of $\si{\ohm}$, $t_{21}$ has unit of $\si{\per\ohm}$
\subsection*{b}
$$
[t] = [t_1][t_2] = 
  \begin{bmatrix}
  \frac{5}{3} & 2\\
  \frac{3}{4} & \frac{3}{2}
  \end{bmatrix}
  \begin{bmatrix}
  1 & 0\\
  \frac{1}{6} & 1
  \end{bmatrix}
$$
$$
= 
  \begin{bmatrix}
  2 & 2\\
  1 & \frac{3}{2}
  \end{bmatrix}
$$

\end{multicols}
\end{document}
