\documentclass[Main.tex]{subfiles}

\begin{document}

\chapter{1}{Introduction and time-domain signal representation}

\section{Signals}
\subsection{Definitions}
\begin{definitions}
\begin{itemize}
    \item Signal: Function of an independent variable, usually time $t$.
    \begin{itemize}
        \item This course will only be about 1D signals, e.g.\ $f(t) = A \cos(\omega t + \theta )$, $t \in \R$.
        \item 2D signals are things like black and white images, i.e.\ $f(x,y)=$ intensity of pixel at $(x, y)$.
    \end{itemize}
    \item Continuous \textbf{Time} Signals: A signal that has a domain $t \in \R$, e.g.\ $f(t)= e^{-t}, t \leq 0$.
    \item Discrete \textbf{Time} Signals: A signal that has a domain $t \in \Z$.
    \item Digital Signals: A discrete signal that has been quantized further in $f(t)$.
    \item Periodic Signals: A signal is periodic when $f(t) \equiv f(t+T)$, where T is as small as possible to satisfy the equation.
    \item Causal Signals:
    \begin{itemize}
        \item Causal: $f(t) = 0, \forall t < 0$
        \item Anticausal: $f(t) = 0, \forall t \geq 0$
        \item Otherwise, noncausal.
    \end{itemize}
\end{itemize}
\end{definitions}
\subsection{Even and Odd signals}
\begin{itemize}
    \item Even: $f(t) = f(-t)$
    \item Odd: $f(t) = -f(t)$
\end{itemize}

\end{document}
