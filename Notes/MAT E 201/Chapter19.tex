\documentclass{article}

\usepackage{mathtools}
\usepackage{tikz}
\usepackage{siunitx}

\newtheorem{theorem}{Example}

\begin{document}
\section*{Electronic Materials}
There are four classes:
\begin{itemize}
    \item Superconductors
          \begin{itemize}
              \item Type 1
              \item Type 2
              \item Ceramic (YBCO)
          \end{itemize}
    \item Conductors
          \begin{itemize}
              \item Linear
              \item Nonlinear
          \end{itemize}
    \item Semi-Conductors
          \begin{itemize}
              \item Elemental
              \item Compound
          \end{itemize}
    \item Dielectrics
          \begin{itemize}
              \item Linear
              \item Nonlinear
          \end{itemize}
\end{itemize}
\subsection*{Electrical Conductivity}

\subsection*{Ohm's Law}
$V = IR$\\
$V$ Voltage \si{\volt}\\
$I$ Current \si{\ampere}\\
$R$ Resistance \si{\ohm}

\subsection*{Power}
\begin{equation*}
    \begin{aligned}
        P           & = VI = I^2 R (\si{\watt}) \\
        V           & = IR                      \\
        R           & = \rho \frac{l}{A}        \\
        V           & = I\rho \frac{l}{A}       \\
        \frac{I}{A} & = \frac{V}{l}\sigma       \\
        \frac{V}{l} & = E \frac{V}{cm}          \\
        \frac{I}{A} & = J \frac{A}{m^2}         \\
        \sigma      & = n \cdot q \cdot \mu
    \end{aligned}
\end{equation*}

\begin{theorem}
    A current of 5A is passed through a 2mm-diameter wire, 500m long, made of
    \begin{itemize}
        \item Copper
        \item Germanium
    \end{itemize}
    Calculate the power loss for Cu and Ge wires
    \begin{equation*}
        \begin{aligned}
            \sigma(Cu) & = 5.98\cdot 10^5 \rho ^-1 cm^-1    \\
            \sigma(Ge) & = 2.10\cdot 10^-2 \rho ^-1 cm^-1   \\
            P=I^2R     & = I^2 \frac{1}{\sigma} \frac{l}{A}
        \end{aligned}
    \end{equation*}
\end{theorem}

\begin{theorem}
    If all the valence electrons in copper contribute to current flow, calculate:
    \begin{itemize}
        \item The mobility of an electronsthe average drift velocity for electrons in a 100cm copper wire when 10v are applied
    \end{itemize}
    \begin{equation*}
        \begin{aligned}
            \sigma & = n \cdot q \cdot \mu          \\
            \mu    & = \frac{\sigma}{nq}            \\
            \sigma & = \num{5.98E5}                 \\
            q      & = \num{1.6E-19}                \\
            a_0    & = \SI{3.6151E-8}{\centi\meter} \\
            \mu &= \SI{44.2}{\centi\meter\squared\per\volt\per\second}
        \end{aligned}
    \end{equation*}
\end{theorem}

\end{document}