\documentclass[12pt]{article}

\usepackage[margin=1.9cm, letterpaper]{geometry}
\usepackage{fontspec}
\usepackage{mathtools}
\usepackage{pgfplots}

\pgfplotsset{compat=newest}

\title{EXPERIMENT 2: \protect\\ Electrostatic Potential \protect\\ and Fields}
\date{February 5, 2019}
\author{Michael Kwok | Partner: Cyrus Diego }

\begin{document}
\maketitle
\pagebreak

\section{Introduction}

In this experiment, we are investigating the effect of Gauss’ laws using a circular and parallel plate capacitor by measuring the potential differences between two conductors in both cases. Due to the relatively simple shapes of the capacitors (two concentric circles and two parallel rectangles), the results can easily be correlated to Gauss’ laws. Since this experiment is conducted with a DC power source as the source of charge, we will be using electrostatic equations which are defined below,

\textbf{Bold} variables represent vectors

\begin{equation} \label{eq:ln_eq}
    \ln r = \ln \frac{A}{B}(\frac{V_r-V_B}{V_0})+\ln B
\end{equation}

where $r$  is the distance of the test point from the center of the circle, $A$
and $B$ are the calculated radii, $V_r$, $V_B$, $V_0$ are the measured voltages.

The equation used to find the strength of the electric field would be:

\begin{equation} \label{eq:ef_strength}
    E=-\frac{\Delta V}{\Delta S}
\end{equation}

\section{Methods}

For part 1, connect the power supply to both leads of the capacitor, and a
voltmeter with the ground side connected to the negative end of the power
supply, and the probe on the appropriate slot of the voltmeter. Record the
voltages at the appropriate distances from the centre of the capacitor, in this
case every 0.5 cm. Adjust precision of voltmeter as necessary. We opted on
recording up to 3 significant digits whenever possible.

For part 2, like the first part, the power supply leads should be connected to
both plates, and the voltmeter with the ground connected to the negative end as
well. Probe the conductive sheet at regular intervals in a grid pattern,
recording the voltage in each spot.

\section{Results}
\begin{center}
\begin{tabular}{ |r|r| }
    \hline
    Radius (cm) & $V_r - V_B$ (V) \\ \hline
    2.5 & 3.55\\
    3.0 & 3.22\\
    3.5 & 4.0\\
    \hline
\end{tabular}
\begin{tabular}{ |r|r| }
    \hline
    $\frac{V_r-V_B}{V_0}$ & $\ln r$ (V) \\ \hline
    2.5 & 3.55\\
    3.0 & 3.22\\
    3.5 & 4.0\\
    \hline
\end{tabular}
\end{center}

% Graph test
\begin{tikzpicture}
    \begin{axis}[
      domain=-1:1,
      samples=101,
      smooth,
      no markers,
      ]
      \addplot {x};
    \end{axis}
  \end{tikzpicture}
  
\section{Discussion}
\section{Conclusion}
\pagebreak
\section*{References}
\pagebreak
\section*{Appendix}
\end{document}