\RequirePackage[l2tabu, orthodox]{nag}
\documentclass{article}

\usepackage[letterpaper, margin=1.3cm]{geometry}
\usepackage{booktabs}
\usepackage{mathtools}
\usepackage[final]{pdfpages}

\title{ECE 487 Assignment 2}
\author{Michael Kwok}
\begin{document}

\maketitle
\subsection*{Detection/Correction}
\begin{itemize}
    \item 6
    \item 7
\end{itemize}

\subsection*{Hamming Distance}
\begin{itemize}
    \item detected: 9, corrected: 4
    \item detected: 6, corrected: 3
\end{itemize}

\subsection*{Decoding}
\begin{itemize}
    \item \(11110 \oplus 11110 = 00000\), hamming distance 0, \verb|11|
    \item \(00010 \oplus 00000 = 00010\), hamming distance 1, \verb|00|
    \item \(00011 \oplus 01011 = 01000\), hamming distance 1, \verb|01|
    \item \(01110 \oplus 11110 = 10000\), hamming distance 1, \verb|11|
\end{itemize}

\subsection*{Linear block code}

This block code is a linear block code. This is due to the fact that every pair of codewords form another valid codeword.

\begin{align*}
    01111 \oplus 10001 & = 11110 \\
    01111 \oplus 11110 & = 10001 \\
    10001 \oplus 11110 & = 01111 \\
\end{align*}

\subsection*{Receiver Action}
\begin{itemize}
    \item Calculate syndrome: \((0 + 1 + 1 + 1 + 1) \mod 2 = 0\), Receiver will return \verb|0111|.
    \item Calculate syndrome: \((0 + 1 + 0 + 1 + 1) \mod 2 = 1\), Receiver will not return data.
\end{itemize}

\subsection*{2D Parity}

\begin{verbatim}
Sent:            Received:
1 1 1 1 1 1      1 0 1 1 0 1
1 0 1 1 1 0      1 0 1 1 1 0
0 1 1 1 0 1  ->  0 1 1 1 0 1
0 1 0 1 0 0      0 1 0 1 0 0
0 1 1 0 0 0      0 1 0 0 1 0
\end{verbatim}
The receiver will calculate the syndrome bits row by row and then column by column. The calculations will look like this:
\begin{itemize}
    \item r1: \((1 + 0 + 1 + 1 + 0 + 1) \mod 2 = 0\)
    \item r2: \((1 + 0 + 1 + 1 + 1 + 0) \mod 2 = 0\)
    \item r3: \((0 + 1 + 1 + 1 + 0 + 1) \mod 2 = 0\)
    \item r4: \((0 + 1 + 0 + 1 + 0 + 0) \mod 2 = 0\)
    \item r5: \((0 + 1 + 0 + 0 + 1 + 1) \mod 2 = 0\)
    \item c1: \((1 + 1 + 0 + 0 + 0) \mod 2 = 0\)
    \item c2: \((0 + 0 + 1 + 1 + 1) \mod 2 = 1\)
    \item c3: \((1 + 1 + 1 + 0 + 0) \mod 2 = 1\)
    \item c4: \((1 + 1 + 1 + 1 + 0) \mod 2 = 0\)
    \item c5: \((0 + 1 + 0 + 0 + 1) \mod 2 = 0\)
    \item c6: \((1 + 0 + 1 + 0 + 0) \mod 2 = 0\)
\end{itemize}

The errors here are detected, but cannot be corrected.
\end{document}
