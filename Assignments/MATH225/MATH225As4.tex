\documentclass{article}

\usepackage[letterpaper, margin=1.3cm]{geometry}
\usepackage[utf8]{inputenc}
\usepackage{siunitx}
\usepackage[fleqn]{mathtools}
\usepackage{amsthm}
\usepackage{amssymb}
\usepackage{mathrsfs}
\usepackage{datetime}
\usepackage{microtype}
\usepackage[l2tabu, orthodox]{nag}

\newcommand\aug{\fboxsep=-\fboxrule\!\!\!\fbox{\strut}\!\!\!}

\title{MATH 225 Assignment 4}
\author{Michael Kwok}
\date{2020-08-01}
\begin{document}
\maketitle
\subsection*{a}
Yes. Each eigenvalue has an algebraic multiplicity of 1.

By using the definition $1 \leq d_{\lambda} \leq m_{\lambda}$:

Since each eigenvalue has $m_{\lambda} = 1, \text{by definition, } d_{\lambda} = 1$.

This shows that each eigenvalue has an eigenspace with basis of size 1, implying the matrix is diagonalizable.

\subsection*{b}
No. Invertible matrices cannot have a 0 as an eigenvalue by the Invertible Matrix Theorem.

\subsection*{c}
The list of eigenvalues is: $\{ 4,4,1,1,0\}$.
Proof:

Let $v_1, v_2, \ldots v_i$ be eigenvectors of A.

Each $v_i$ related to an eigenvalue $\lambda$ from the list.
\begin{align*}
    A v_i &= \lambda_i v_i\\
    A^2 v_i &= AA v_i\\
    &= A \lambda_i v_i\\
    &= \lambda_i A v_i\\
    &= \lambda^2_i v_i
\end{align*}

\subsection*{d}

Yes it is diagonalizable.
\begin{align*}
    A^2 &= AA\\
    &= PDP^{-1}PDP^{-1}\\
    &= P DD P^{-1}\\
    &= P D^2 P^{-1}
\end{align*}
\end{document}