\documentclass{article}

\usepackage[letterpaper, margin=1.3cm]{geometry}
\usepackage[utf8]{inputenc}
\usepackage{siunitx}
\usepackage[fleqn]{mathtools}
\usepackage{amssymb}
\usepackage{mathrsfs}

\title{MATH 225 Assignment 1}
\author{Michael Kwok}
\begin{document}
\maketitle
\subsection*{1}
Let $\mathbf{u} \in V$

Suppose there exists $w_1, w_2 \in V$ such that $\mathbf{u} \oplus \mathbf{w_1} = \Tilde{\mathbf{z}}$ and $\mathbf{u} \oplus \mathbf{w_2} = \Tilde{\mathbf{z}}$.

1. $\mathbf{u}\oplus \mathbf{w_1} = \Tilde{\mathbf{z}}$

2. $\mathbf{u}\oplus \mathbf{w_2} = \Tilde{\mathbf{z}}$

3. $\mathbf{w_1} = - \mathbf{u}$ Statement 1 by axiom 5

4. $\mathbf{w_2} = - \mathbf{u}$ Statement 2 by axiom 5

$\therefore \mathbf{w_1} = \mathbf{w_2}$ Statement 3 and 4
\newpage
\subsection*{2a}
Axiom 1:

$\Tilde{\mathbf{z}} = (0, 0)$

$(x, y) = (0, 0)$

$0^2 + 0^2 \leq 1$

\noindent
Axiom 2:

Counter example

Let $\mathbf{u} = \left(0.5, 0.5\right)$

Let $\mathbf{v} = \left(0.5, 0.4\right)$

$\mathbf{u} + \mathbf{v} = (1, 0.9)$

$1^2 + 0.9^2 = 1.81 > 1$

$(1, 0.9) \not\in H$

It does not fulfil this axiom


$\therefore$ It is not a subspace.

\subsection*{2b}
Axiom 1:

From $\mathscr{P}_{n=0}, \Tilde{\mathbf{z}} = 0$

When $n = 0 \text{ and } a_0 = 0$, $0 \in H$

$\therefore \Tilde{\mathbf{z}} \in H$

\noindent
Axiom 2:

let $\mathbf{u} = a_0 + a_1x + \cdots + a_nx^n$

let $\mathbf{v} = b_0 + b_1x + \cdots + b_nx^n$
\begin{align*}
\mathbf{u} + \mathbf{v} &= (a_0 + a_1x + \cdots + a_nx^n) + (b_0 + b_1x + \cdots + b_nx^n)\\
&= (a_0 + b_0) + (a_1 + b_1)x + \cdots + (a_n + b_n)x^n
\end{align*}
\begin{align*}
\left(a_0 - a_1 + ... + (-1)^n a_n\right) = 0\\
\left(b_0 - b_1 + ... + (-1)^n b_n\right) = 0\\
\end{align*}
\noindent
Axiom 3:
\begin{align*}
\text{Let } c\mathbf{u} &= c\left(a_0 + a_1x + \cdots + a_n x^n\right)\\
&= ca_0 + ca_1x \cdots + ca_nx^n
\end{align*}
\begin{align*}
\left[ca_0 - ca_1 + \cdots +  c(-1)^na^n\right] &= 0\\
c\left[a_0 - a_1 + \cdots + (-1)^n a^n\right]&=0\\
\end{align*}
$\therefore$ It is a subspace

\subsection*{2c}
Use SV2 to show $W$ is a valid subspace.

Let $k_1, k_2 \in \mathbb{R}$

Let $\mathbf{u} = \begin{bmatrix}
a_1 + 2b_1 & b_1 + c_1\\
b_1 - c_1 & a_1 + 4c_1
\end{bmatrix}$

Let $\mathbf{v} = \begin{bmatrix}
a_2 + 2b_2 & b_2 + c_2\\
b_2 - c_2 & a_2 + 4c_2
\end{bmatrix}$

Show that $k_1 \mathbf{u} + k_2 \mathbf{v} \in W$
\begin{align*}
k_1 \mathbf{u} + k_2 \mathbf{v} &=
\begin{bmatrix}
k_1a_1 + 2 k_1 b_1 + k_2 a_2 + 2 k_2 b_2 & k_1 b_1 + k_1 c_1 + k_2 b_2 + k_2 c_2\\
k_1 b_1 - k_1 c_1 + k_2 b_2 - k_2 c_2 & k_1 a_1 + 4 k_1 c_1 + k_2 a_2 + 4 k_2 c_2
\end{bmatrix}\\
&= \begin{bmatrix}
(k_1 a_1 + k_2 a_2) + 2 (k_1 b_1 + k_2 b_2) & (k_1 b_1 + k_2 b_2) + (k_1 c_1  + k_2 c_2)\\
(k_1 b_1 + k_2 b_2) - (k_1 c_1 + k_2 c_2) & (k_1 a_1 + k_2 a_2) + 4 (k_1 c_1  + k_2 c_2)
\end{bmatrix}
\end{align*}
$a= k_1 a_1 + k_2 a_2,\text{ }b=k_1 b_2 + k_2 b_2,\text{ }c=k_1 c_1 + k_2 c_2$

$a, b, c \in \mathbb{R}$

$k_1 \mathbf{u} + k_2 \mathbf{v} \in W$

$\therefore W$ is a subspace of $M_{2,2}$
\newpage
\subsection*{3a}
\noindent
Let $\mathcal{B} = \left\{
\begin{bmatrix}
1 & 0\\
0 & 0
\end{bmatrix}, 
\begin{bmatrix}
0 & 1\\
1 & 0
\end{bmatrix},
\begin{bmatrix}
0 & 0\\
0 & 1
\end{bmatrix}\right\}$

\noindent
B1:

$a \begin{bmatrix}
1 & 0\\
0 & 0
\end{bmatrix} + 
b \begin{bmatrix}
0 & 1\\
1 & 0
\end{bmatrix} +
c \begin{bmatrix}
0 & 0\\
0 & 1
\end{bmatrix} = 
\begin{bmatrix}
0 & 0\\
0 & 0
\end{bmatrix} \iff a, b, c = 0$


$\therefore \mathcal{B}$ is linearly independent.

\noindent
B2:

$S_{2,2} = \left\{
\begin{bmatrix}
a & b\\
b & c
\end{bmatrix}: a, b, c \in \mathbb{R}
\right\}$

$a \begin{bmatrix}
1 & 0\\
0 & 0
\end{bmatrix} + 
b \begin{bmatrix}
0 & 1\\
1 & 0
\end{bmatrix} +
c \begin{bmatrix}
0 & 0\\
0 & 1
\end{bmatrix} = 
\begin{bmatrix}
a & b\\
b & c
\end{bmatrix}$

$\therefore \mathcal{B} \text{ spans } V$


\subsection*{3b}

Let $\mathcal{B} = \left\{1\right\}$

\noindent
B1:

$a \cdot 1 = 0 \iff a = 0$

\noindent
B2:

Let $b \in \left( \mathbb{R}_{>0}, \times,\wedge \right)$
\begin{align*}
    a \cdot 1 &= b\\
    a &= b
\end{align*}

For any value $b \in \left( \mathbb{R}_{>0}, \times,\wedge \right)$, the basis could be used to form a solution.


\end{document}