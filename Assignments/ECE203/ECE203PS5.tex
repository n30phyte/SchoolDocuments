\documentclass{article}

\usepackage[letterpaper, margin=1.3cm]{geometry}
\usepackage[utf8]{inputenc}
\usepackage{siunitx}
\usepackage{mathtools}
\usepackage{multicol}
\usepackage{pgfplots}
\usepackage[RPvoltages, american, betterproportions, siunitx]{circuitikz}
\pgfplotsset{compat=1.16}
\pgfplotsset{scaled x ticks=true}

\title{ECE 203 Problem Set 5}
\author{Michael Kwok}
\date{March 2020}
\begin{document}

\maketitle
\begin{multicols}{2}
\section*{1}
\subsection*{a}

\begin{align*}
    \omega_0 &= \frac{1}{\sqrt{L C}} = \boxed{\SI{1390.77}{\radian\per\second}}\\
    f_0 &= \frac{\omega_0}{2\pi} = \boxed{\SI{221.3}{\hertz}}\\
    Q_0 &= \omega_0 R C\\
    &= 1390.77 \cdot 1000 \cdot \num{47d-6} = \boxed{65.37}
\end{align*}

\subsection*{b}

\begin{align*}
    \omega_0 \text{ parallel} &= \omega_0 \text{ series}\\
    f_0 \text{ parallel} &= f_0 \text{ series}\\
    Q_0 \text{ series} &= \frac{1}{Q_0 \text{ parallel}}\\
\end{align*}
\begin{align*}
    \omega_0 &= \boxed{\SI{1390.77}{\radian\per\second}}\\
    f_0 &= \boxed{\SI{221.3}{\hertz}}\\
    Q_0 &= \boxed{0.0153}
\end{align*}

\subsection*{c}

The parallel configuration is better as the quality factor is higher, which by definition means that the filter is more selective.

\section*{2}
\subsection*{a}

\begin{align*}
    Q_0 &= \omega_0 R C\\
    \omega_0 &= 2\pi f_0 \\
    Q_0 &= 2\pi f_0 R C\\
\end{align*}

\begin{align*}
    R  &= \frac{V}{I}\\
       &= \frac{\num{125e-3}}{2e-6}\\
       &= \boxed{\SI{62500}{\ohm}}\\
    RC &= \frac{40}{2\pi \cdot 1000000}\\
    C  &= \frac{\num{6.366197724e-6}}{62500}\\
       &= \SI{1.018591636e-10}{\farad}\\
       &= \boxed{\SI{102}{\pico\farad}}
\end{align*}
\begin{align*}
    \omega_0 &= \frac{1}{\sqrt{LC}}\\
    \num{2\pi e-6} &= \frac{1}{\sqrt{LC}}\\
    LC &= \left(\frac{1}{\num{2\pi e-6}}\right)^2\\
    L &= \boxed{\num{2.5e-4}}
\end{align*}

\subsection*{b}
\begin{align*}
\text{Using Z at } \omega = 2000,\\
R= \boxed{\SI{8}{\kilo\ohm}}
\end{align*}
\begin{align*}
    Z&=\left( \frac{1}{R} + \frac{1}{j \omega L} + j \omega C \right) ^{-1}\\
    \frac{1}{4000} e^{-j \theta} &= \frac{1}{8000} + \frac{1}{1800 L j} + 1800 C j\\
    \frac{1}{4000} \cos \theta &= \frac{1}{8000}\\
    \cos\theta &= \frac{1}{2}\\
    \theta &= \frac{\pi}{3}
\end{align*}
\begin{align*}
    -j\frac{1}{4000} \sin\theta &= -j\left( \frac{1}{1800L} - 1800 C\right)\\
    \frac{1800L}{4000} \sin\theta &= -(1800)^2 LC + 1\\
    \omega_0 &= 2000\\
    &= \frac{1}{\sqrt{LC}}\\
    LC&= \frac{1}{\num{4e6}}\\
\end{align*}
\begin{align*}
    L\sin\theta &= 0.4222\\
    L &= \boxed{\SI{0.488}{\henry}}\\
    C &= \boxed{\SI{5.133e-7}{\farad}}
\end{align*}

\section*{3}
\subsection*{a}
\begin{align*}
    Z_{in} &= 10 + j\omega 3 + \left( \frac{1}{75} + \frac{1}{15 + \frac{675}{j\omega}} \right) ^{-1}\\
    &= \frac{90\omega + (6 \omega^2 - 1275)j}{2\omega - 15j}\\
    &= \frac{\left[ 90\omega + (6 \omega^2 - 1275)j \right] [2\omega + 15j]}{4\omega^2 + 15^2}\\
    &= \frac{90\omega^2 +19125 + \left( 12\omega^3 -1200 \omega j \right)}{4\omega^2 + 15^2}
\end{align*}
\begin{align*}
    12\omega^3-1200\omega &= 0\\
    12\omega^3 &= 1200\omega\\
    \omega^2 &= 100\\
    \omega &= \boxed{\SI{10}{\radian\per\second}}
\end{align*}

\subsection*{b}
\begin{align*}
    Z_{in} &= 10 + 30j + \left( \frac{1}{75} + \frac{1}{15 + \frac{675}{j10}} \right)^{-1}\\
    &= 10 + 30j - 30j + 35\\
    &= \boxed{\SI{45}{\ohm}}
\end{align*}

\section*{4}
Use a test voltage $V_{in}$

\begin{circuitikz}
\draw (0,0) to[V=$V_{in}$] (0,2) -- (4,2);
\draw (0,0) -- (7,0) to[cV] (7,2);
\draw (2,2) to[C, i=$I_C$](2,0);
\draw (4,0) to[R, i=$I_R$](4,2);
\draw (4,2) to[L] (7,2);
\end{circuitikz}

\begin{align*}
    I_C &= V_{in}\num{e-8}j\omega\\
    I_R &= -\frac{V_{in}}{10000}\\
\end{align*}
    Using KCL on node above resistor,
\begin{align*}
    -I_C + I_R &+\frac{V_{in} - \num{e5}I_R}{\num{4.4e-3}j\omega } = 0\\
    -V_{in}\num{e-8}j\omega-\frac{V_{in}}{10000} &+\frac{V_{in} - \num{e5}I_R}{\num{4.4e-3}j\omega} = 0\\
    Y = &\frac{V_{in}\num{e-8}j\omega +\frac{V_{in}}{10000} +\frac{V_{in} \left( 1+\frac{\num{e5}}{10000}\right)}{\num{4.4e-3}j\omega}}{V_{in}}\\
    = &\boxed{\num{e-8}j\omega +\frac{1}{10000} -\frac{2500}{\omega}j}
\end{align*}
Finding $\omega_0$ and $Q_0$.

$\omega_0$ is when coefficient of $j = 0$
\begin{align*}
    \text{Start with: }\num{e-8}j\omega&+\frac{1}{10000} -\frac{2500}{\omega}j\\
    \num{e-8}j\omega &-\frac{2500}{\omega}j = 0\\
    \omega^2 &= \num{2.5e4}\\
    \omega &= \boxed{\SI{500}{\kilo\radian}}\\
    Q_0 &= \omega_0 R C\\
    &\boxed{= 50}
\end{align*}
\section*{5}
\subsection*{a}
$H_{dB} = 20 \log(|\tilde{H}(j\omega)|)$
\begin{enumerate}
    \item $20 \log(0.2) = \boxed{-13.979}$
    \item $20 \log(13) = \boxed{22.3}$
    \item $20 \log(|\frac{100}{23j +21}|) = \log(|\frac{2300 j - 2100}{23^2+21^2}|) = 20 \log(3.21) = \boxed{10.13}$
\end{enumerate}
\subsection*{b}
$|\tilde{H}(j\omega)| = 10^{\frac{H_{dB}}{20}}$
\begin{enumerate}
    \item $|\tilde{H}(j\omega)| = \boxed{39.8}$
    \item $|\tilde{H}(j\omega)| = \boxed{0.447}$
    \item $|\tilde{H}(j\omega)| = \boxed{1.00}$
\end{enumerate}
\end{multicols}
\subsection*{c}
\begin{align*}
    \tilde{H}(j\omega) &= \frac{V_{out}}{V_{in}}\\
    Z &= 1+ \frac{1}{j\omega \num{100e-6}}\\
    \tilde{H}(j\omega) &= \frac{1}{1 + j\omega\num{100e-6}}\\
    H_{dB} &= 20 \log(|\tilde{H}(j\omega)|) = 20\log\left(\frac{1}{\num{e-8}\omega^{2} + 1}\right)\\
    \angle\tilde{H}(j\omega) &= -\arctan\left(0.0001\omega\right)
\end{align*}
\begin{figure}
    \centering
\begin{tikzpicture}
\begin{axis}[xmode=log,
xmin=1,
xmax=1000000,
ymin=-40,
ymax=10,
grid=both]
\addplot table [smooth, x={omega}, y={result}, col sep=comma, mark=none]{res1.csv};
\addplot[gray,ultra thick, opacity=0.6] coordinates {
(1,0)
(10000,0)
(99500, -40)};
\end{axis}
\end{tikzpicture}
    \caption{Magnitude bode plot}
    \label{fig:my_label}
\end{figure}
\begin{figure}
    \centering
\begin{tikzpicture}
\begin{axis}[xmode=log,
xmin=1,
xmax=1000000,
ymin=-135,
ymax=45,
ytick={45,0,-45,-90,-135},
grid=both]
\addplot table [smooth, x={omega}, y={atan}, col sep=comma, mark=none]{res1.csv};
\addplot[gray,ultra thick, opacity=0.6] coordinates {
(1,0)
(1000,0)
(100000,-90)
(1000000, -90)};
\end{axis}
\end{tikzpicture}
    \caption{Phase bode plot}
    \label{fig:label2}
\end{figure}
\end{document}
