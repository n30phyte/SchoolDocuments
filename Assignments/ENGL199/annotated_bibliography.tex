\documentclass[11pt]{article}
\usepackage[margin=1in, letterpaper]{geometry}
\usepackage[style=ieee, backend=biber]{biblatex}

\addbibresource{citations.bib}

\title{Annotated Bibliography \\ of \\ Speculative Execution \\ Processor Vulnerabilities}
\author{Michael Kwok (mkwok1@ualberta.ca) \\ University of Alberta \\ Edmonton, Alberta}
\date{\today}

\begin{document}
\pagestyle{headings}
\maketitle

\cite{Meltdown} \fullcite{Meltdown}

The authors of the paper are the computer security researchers who discovered
the Meltdown exploit, and in the whitepaper provide sample outputs that show
the exploit in action. The paper points out one feature of modern CPUs that allow
for the vulnerability to exist, which is called "Out-of-order execution". This
feature allows CPUs to cut down on program execution time by running code that
they think will evenetually get run in the future, and execute them ahead of
time. The researchers claimed to
be able to read the kernel's at the speed of 503KB/s, which is quite quick for an
attack that heavily relies on trial and error. This paper gives an interesting
implcation to the disaster, as this affected one company's (Intel) CPUs
disproportionately more than their competitors' (AMD, ARM) CPUs, which could
point to either an engineering or management problem inherent in that company.
Since the paper is a direct source from the researchers, the information can be
taken as correct and accurate. It also contains many references to prior works
on exploiting this specific feature, but most of the previous research were only
theoretical in nature, and were not exploitable, which makes this discovery
novel at the time. However, the paper does not provide sample code that show a
full implementation of the problem, which makes it difficult for beginning
researchers to fully understand the exploit. Although this makes sense considering
the unfixable nature of the exploit, and the not wanting to allow the code to be
reused for nefarious purposes, a benign example would still be appreciated, especially
to prove the claims of the high read speeds.

% TODO: Reread and delete unnecessary things (Too specific?)
\hfill

\cite{Spectre} \fullcite{Spectre}

Similar to the previous paper, this was written by the researchers who
discovered the exploit. In fact, there are some overlapping authors between the
papers. This exploit is much less easily fixed compared to the previous one, and
in this paper, the authors described different ways of "Side channel attacks"
that have been attempted before. 

% TODO: Talk more about Spectre

\end{document}