\RequirePackage[l2tabu, orthodox]{nag}
\documentclass{article}

\usepackage[letterpaper, margin=1.3cm]{geometry}
\usepackage{siunitx}
\usepackage{mathtools}
\usepackage{multicol}
\usepackage{amssymb}
\usepackage{mathrsfs}
\usepackage{enumitem}
\usepackage{circuitikz}
\usepackage{booktabs}
\usepackage{multirow}
\usepackage[outputdir=obj]{minted}

\ctikzset{
    logic ports=ieee,
    logic ports/scale=0.7,
}

\title{ECE 410 Assignment 1}
\author{Michael Kwok}

\begin{document}
\maketitle
\begin{enumerate}
    \item

    \item
          Code:
          \begin{minted}{vhdl}
        TYPE quark_t IS (up, down, charm, strange, top, bottom);
      \end{minted}

          \begin{enumerate}
              \item 3
              \item down
              \item strange
              \item top
              \item strange
              \item charm
          \end{enumerate}

    \item `U' means the value is not initialized yet, so it could be anything from a strong signal or a floating value, as we have not assigned anything to it yet. `X' is the value that signifies multiple drivers are assigning values to that port, indicating that there might be an issue with the design. `W' is for weak signal where it can't be determined whether it's a weak high or low. `Z' is a high impedance value used for implementing three-state logic, ``disconnecting'' that port from the bus. `-' is for don't care values, usually for control flow and other matching statements.

    \item Code:
    \begin{minted}{vhdl}
        TYPE quark_t IS (up, down, charm, strange, top, bottom);
      \end{minted}

\end{enumerate}
\end{document}
