\documentclass[landscape, letterpaper, 8pt]{extarticle}

\usepackage[margin=.25in]{geometry}
\usepackage{amssymb}
\usepackage{mathtools}
\usepackage{multicol}
\usepackage{enumitem}
\usepackage{scrextend}

\renewcommand{\complement}[1]{{#1}^\mathsf{c}}
\DeclarePairedDelimiter{\floor}{\lfloor}{\rfloor}

\begin{document}
\begin{multicols}{3}
    \section*{Logic Properties}
    \begin{itemize}[noitemsep,nolistsep]
        \item \textbf{Generalization: } $p~\therefore p \lor q$ and $q~\therefore p \lor q$
        \item \textbf{Specialization: } $p \land q~\therefore p$ or $q$
        \item \textbf{Conjunction: } $p,~q~\therefore p \land q$
        \item \textbf{Eliminiation: } $p \lor q,~\neg q~\therefore p$
        \item \textbf{Division into Cases: } $p \lor q,~p\implies r,~q\implies r~\therefore r$
        \item \textbf{Contradiction Rule: } $\neg p \implies c~\therefore p$
    \end{itemize}
    \subsection*{Implies}
    \begin{itemize}[noitemsep,nolistsep]
        \item \textbf{Statement: } $p \implies q$
        \item \textbf{Converse: } $q \implies p$
        \item \textbf{Inverse: } $\neg p \implies \neg q$
        \item \textbf{Contrapositive: } $\neg q \implies \neg p$
    \end{itemize}
    \subsection*{Quantified}
    \begin{itemize}[noitemsep,nolistsep]
        \item \textbf{Universal Instantiation: } $\forall x,~P(x)~\therefore P(a)$ also applies to $\exists$.
        \item \textbf{Universal Generalization: }$P(a)~\therefore \forall x,~P(x)$ also applies to $\exists$.
        \item \textbf{Negation: }$\neg(\forall x, Q(X)) \equiv \exists x, \neg Q(x)$ and $\neg(\exists x, Q(X)) \equiv \forall x, \neg Q(x)$.
        \item \textbf{Universal Modus Ponens: } $\forall x, P(x)\implies Q(x),~P(a)~\therefore Q(a)$ does not apply to $\exists$.
        \item \textbf{Universal Modus Tolens: } $\forall x, P(x)\implies Q(x),~\neg Q(a) \therefore \neg P(a)$ does not apply to $\exists$.
        \item \textbf{Converse Error: }$ \begin{aligned}
                                 & \forall x, P(x) \implies Q(x) \\
                                 & Q(a)                          \\
                      \therefore & P(a)
                  \end{aligned}$
        \item \textbf{Inverse Error: } $\begin{aligned}
                                 & \forall x, P(x) \implies Q(x) \\
                                 & \neg P(a)                     \\
                      \therefore & Q(a)
                  \end{aligned}$

    \end{itemize}
    \section*{Induction}
    \subsection*{Strong Induction}
    \textbf{Let} $P(n)$ be the property to be proven, let $a$ and $b$ be fixed integers where $a\leq b$.

    Suppose the following:
    \begin{enumerate}[noitemsep,nolistsep]
        \item \textbf{Basis:} $P(a), P(a+1), \ldots P(b)$ are true.
        \item \textbf{Induction:} For all integers $k\geq b$, if $P(i)$ is true for all integers from $a$ to $k$ then $P(k+1)$ is true.
        \item Hence for all integers $n\geq a, P(n)$ is true.
    \end{enumerate}
    \section*{Number Theory}
    \begin{itemize}[noitemsep,nolistsep]
        \item \textbf{Even} $n$ is even $\iff \exists k, n = 2k$.
        \item \textbf{Odd} $n$ is odd $\iff \exists k, n = 2k+1$.
        \item \textbf{Prime} $n$ is prime $\iff \forall r, s$, if $n = rs$ then $r=1$ and $s = n$ or vice versa.
        \item \textbf{Composite} $n$ is composite $\iff \exists r, s$, if $n = rs$ and $1 < r < s$ and $1 < s < n$.
        \item \textbf{Rational} $r$ is rational $\iff \exists a, b$ such that $r = \frac{a}{b}$ and $b \neq 0$.
        \item \textbf{Divisible} $d|n \iff \exists k$ such that $n=dk$.
    \end{itemize}
    \section*{Sets}
    Can be proven by using Induction, Algebra or Element method.

    Element Method: Usually with equality or subset relations: prove that an element in LHS is an element in RHS.
    \begin{itemize}[noitemsep,nolistsep]
        \item Proper Subset: $A \subseteq B$ \textbf{AND} there is one element in B that is \textbf{not} in A
        \item Power Set: $\wp(\{x, y, z\}) = \{\emptyset, \{x\}, \{y\}, \{z\},\\
                  \{x, y\},\{x, z\}, \{y, z\}, \{x, y, z\}$
        \item Disjoint set:$A\cap B = \varnothing$
    \end{itemize}
    \subsection*{Identities}
    \begin{itemize}[noitemsep,nolistsep]
        \item Commutative Law
              \begin{itemize}[noitemsep,nolistsep]
                  \item $A \cup B = B \cup A $ and $A \cap B = B \cap A$
              \end{itemize}
        \item Associative Law
              \begin{itemize}[noitemsep,nolistsep]
                  \item $(A \cup B) \cup C = A \cup (B \cup C)$ and $(A \cap B) \cap C = A \cap (B \cap C)$
              \end{itemize}
        \item Distributive Law
              \begin{itemize}[noitemsep,nolistsep]
                  \item $A \cup (B \cap C) = (A \cup B) \cap (A \cup C)$
                  \item $A \cap (B \cup C) = (A \cap B) \cup (A \cap C)$
              \end{itemize}
        \item Identity Law
              \begin{itemize}[noitemsep,nolistsep]
                  \item $A \cup \varnothing = A$ and $A \cap U$
              \end{itemize}
        \item Idempotent Law
              \begin{itemize}[noitemsep,nolistsep]
                  \item $A \cup A = A$ and $A\cap A = A$
              \end{itemize}
        \item Universal Bound Law
              \begin{itemize}[noitemsep,nolistsep]
                  \item $A \cup U = U$ and $A  \cap \varnothing = \varnothing$
              \end{itemize}
        \item DeMorgan's Law
              \begin{itemize}[noitemsep,nolistsep]
                  \item $\complement{(A \cup B)} = \complement{A} \cap \complement{B}$ and $\complement{(A \cap B)} = \complement{A} \cup \complement{B}$
              \end{itemize}
        \item Absorption Law
              \begin{itemize}[noitemsep,nolistsep]
                  \item $A \cup (A \cap B) = A$ and $A \cap (A \cup B) = A$
              \end{itemize}
        \item Set Difference Law
              \begin{itemize}[noitemsep,nolistsep]
                  \item $A-B = A \cap \complement{B}$
              \end{itemize}
        \item Complement Law
              \begin{itemize}[noitemsep,nolistsep]
                  \item $A\cup \complement{A} = U$ and $A \cap \complement{A} = \varnothing$
              \end{itemize}
    \end{itemize}
    \section*{Functions}
    A function $f$ from a set $X$ to a set $Y$ denoted by $f: X \to Y$ where $X$ is the domain and $Y$ is the co-domain.
    Every element in $X$ relates to some element in $Y$ and each element in $X$ must map to at most one element in $Y$.
    \subsection*{One-to-One functions}
    $\forall x_1, x_2 \in X[(F(x_1) = F(x_2)) \implies (x_1 = x_2)]$
    \subsection*{Onto functions}
    $\forall y \in Y[ \exists x \in X \text{ where } F(x)=y]$
    \subsection*{Inverse functions}
    $F^{-1}(y) = x\text{ where }F(x) = y$.
    $F(x)$ has to be bijective (both one-to-one and onto).
    \section*{Relations}
    A partition of set $A$ is a collection of nonempty, mutually disjoint subsets whose union is $A$, i.e. $A_1 \cup A_2 \cup A_3 \cup \ldots \cup A_i = A$
    \begin{itemize}[noitemsep,nolistsep]
        \item Reflexive $\forall x \in A[(x, x) \in \mathbb{R}]$
        \item Symmetric $\forall x, y \in A[(x, y) \in \mathbb{R} \implies (y, x) \in \mathbb{R}]$
        \item Transitive $\forall x, y, z \in A[(x R y \land y R z)\implies x R z]$
        \item Partition-induced relation: $\forall x, y \in A[x R y \iff (x, y) \in \text{single partition }A_i]$
        \item Equivalence Relation: When a relation is relation is reflexive, symmetric and transitive.
        \item Identity Relation: $I_A = \{(x, y) \in A \times A~|~x=y\}$
    \end{itemize}
    \subsection*{Equivalence Classes}
    The subset of all elements that are related to $a$.

    Suppose $A$ is a set and $R$ is an equivalence relation on $A$. For each element $a$ in $A$, the \textbf{equivalence class of $a$}, $[a]$ or \textbf{class of $a$} is the set of all elements $x$ in $A$ such that $x$ is related to $a$ by $R$. $$[a] = \{x \in A~|~x~R~a\}$$
    \section*{Partial Order Relations}
    A relation that is reflexive, antisymmetric and transitive.
    \subsection*{Antisymmetry}
    A relation is antisymmetric if $x R y \land y R x,~x = y$
    A relation is \textbf{not} antisymmetric if $x R y \land y R x\text{ but }x \neq y$
    \section*{Counting}
    $P(E)$ is $$ P(E) = |E|/|S|$$ where S is the set of possible outcomes and E is the set of events.

    Number of elements: $|{n, n+1, n+2, \ldots, m}| = m - n + 1$
    \subsection*{Permutations and Combinations}
    Number of possible arrangements of n units: $n!$
    \begin{itemize}[noitemsep,nolistsep]
        \item With $n_1$ duplicates: $\frac{n!}{n_1!}$
        \item With $m$ units in certain order: $(n-(m-1))!$
        \item In a circle/rotation: $\frac{n!}{n}$ (Let 1 be fixed, arrange the rest.)
    \end{itemize}

    $P(n, r) = \frac{n!}{(n-r)!}$

    $C(n, r) = \binom{n}{r} = \frac{P(n, r)}{r!}$

    Number of Combinations with repetitions allowed: $\binom{r+n-1}{r}$

    $\binom{n}{r} = \binom{n}{n-r}$

    Pascal's Formula: $\binom{n+1}{r} = \binom{n}{r-1}+ \binom{n}{r}$

    Binomial Theorem:\begin{align*}
        (a+b)^n & = \sum_{k=0}^n\binom{n}{k} a^{n-k}b^k                                  \\
                & = a^n + \binom{n}{1} a^{n-1} b^1                                       \\
                & + \binom{n}{2} a^{n-2} b^2 + \ldots + \binom{n}{n-1} a^1 b^{n-1} + b^n \\
    \end{align*}
    \subsection*{Pigeonhole Principle}
    If $n$ pigeons fly into $m$ pigeonholes, and $n > km$ for some $k \in \mathbb{Z}^+$ then at least one pigeonhole contains k+1 or more pigeons.

    Generalized: $$k = \floor*{\frac{n}{m}}$$

    "The maximum value has at least the average value, for any non-empty finite bag of real numbers" -Prof. Djikstra
    \section*{Examples}
    \newtheorem{example}{Example}
    \subsection*{Strong Induction}
    \begin{example}
        \begin{equation*}
            \begin{aligned}
                f(n) = \begin{cases}
                    1                   & \text{if }n=1      \\
                    3                   & \text{if }n=2      \\
                    f(n-1) + 2f(n-2) +3 & \text{if } n\geq 3
                \end{cases}
            \end{aligned}
        \end{equation*}
        Using mathematical induction, prove that for all $n \geq 3: f(n) \geq 2^n$

        Answer:
        \begin{equation*}
            \begin{aligned}
                f(k+1) =f(k)+ 2f(k-1)+3              \\
                \text{Use fact that $f(n) \geq 2^n$} \\
                f(k+1) \geq 2^k+2\cdot 2^{k-1} +3    \\
                \text{and } f(k+1) \geq 2^k + 2^k +3 = 2^{k+1} + 3 \geq 2^{k+1}
            \end{aligned}
        \end{equation*}
    \end{example}
    \subsection*{Weak Induction}
    \begin{example}
        What is the smallest value of k such that any integer postage of k, or
        more, cents can be formed by using only 3-cent and 8-cent stamps?

        Use mathematical induction to prove that all integer postage of k, or more, cents can be
        obtained.
        \begin{equation*}
            \begin{aligned}
                \textbf{Basis: } P(n):                             & n = 3a + 8b       \\
                \textbf{Induction: } P(n+1):                       & n+1 = 3a' + 8b'   \\
                \textbf{Case 1: } \text{if } b \geq 1, n = 3a + 8b & \text{, so}       \\
                n+1 = 3a+1+8b                                      & = 8(b-1) + 9 + 3a \\
                n+1                                                & = 8(b-1) +3(a+3)  \\
                \textbf{Hence }a' = a+3      \text{ and }          & b' = b - 1        \\
                \textbf{Case 2: } \text{if } b = 0, n = 3a         &                   \\
                \text{Since }n \geq 14,~a \geq 5                                       \\
                n+1 = 1 + 15 + 3(l-5)                              & = 16+ 3(l-5)      \\
                \textbf{Hence }a' = l-5      \text{ and }          & b' = 2            \\
            \end{aligned}
        \end{equation*}
    \end{example}
    \begin{example}
        Recall that the $n$'th Harmonic number, $H_n$, is defined as:
        $$
            H_n = 1+ \frac{1}{2} + \frac{1}{3} + \ldots + \frac{1}{n}
        $$

        Use mathematical induction to prove that for all positive integers $n$:

        $$
            H_1 + H_2 + \ldots + H_n = (n+1)H_n -n
        $$

        Answer:
        \begin{equation*}
            \begin{aligned}
                \textbf{Basis: } P(n): & H_1 + H_2 + \ldots + H_n = (n+1) H_n-n \\
                P(1):                  & H_1 = 1 = 2H_1 -1                      \\
            \end{aligned}
        \end{equation*}
        \begin{equation*}
            \begin{aligned}
                \textbf{Induction: } & H_1 + \ldots + H_k + H_{k+1} = (k+1)H_k - k + H_{k+1}                  \\
                \text{Using IH for k}                                                                         \\
                =                    & (k+1)(1+\frac{1}{2}+\ldots+\frac{1}{k})-k  +(1+\ldots + \frac{1}{k+1}) \\
                =                    & \ldots                                                                 \\
                % Rest of the fucking owl
                =                    & (k+2) (1+\frac{1}{2} + \ldots + \frac{1}{k+1}) - (k+1)
            \end{aligned}
        \end{equation*}
    \end{example}
    \subsection*{Functions}
    Let $f: X \mapsto Y$ and $g: Y \mapsto Z$ be functions. Indicate whether each statement below is true or false. Give a proof for each true statement and a counterexample for each false statement.

    \textbf{a) If $g \circ f$ is onto then $f$ is onto.}

    \textbf{Answer: } False.

    $X = {1, 2}, Y = {1, 2, 3}, Z = {1, 2}, f(1) = 1, f(2) = 2, g(1) = 1, g(2) = 2, g(3) = 2$

    \textbf{b) If $g \circ f$ is onto then $g$ is onto.}

    \textbf{Answer: } True.

    Let $z \in Z$. There exists $x \in X$ such that $g(f(x)) = z$ since $g \circ f$ is onto. So $g(y) = z$ for $y = f(x)$. Therefore $g$ is onto.

    \textbf{c) If $g \circ f$ is one-to-one then $f$ is one-to-one.}

    \textbf{Answer: } True.

    Let $x_1, x_2 \in X$ such that $f(x_1) = f(x_2)$.
    
    Then $g(f(x_1)) = g(f(x_2))$. Now $g(f(x_1)) = g(f(x_2))$ by definition of composition. Therefore $x_1 = x_2$ since $g \circ f$ is one-to-one. This implies $f$ is one-to-one.

    \textbf{d) If $g \circ f$ is one-to-one then $g$ is one-to-one.}

    \textbf{Answer: } False.

    Use part (a).

    \subsection*{Counting}
    \begin{example}
        How many three-digit integers are divisible by 5?
        \begin{equation*}
            \begin{aligned}
                \text{Three-digit integers} = {100, 101, \ldots, 999}                                         \\
                \left\lfloor \frac{100}{5} \right\rfloor = 20, \left\lfloor \frac{999}{5} \right\rfloor = 199 \\
                \text{by number of elements theorem: } 199-20 + 1 = 180
            \end{aligned}
        \end{equation*}
    \end{example}
    \begin{example}
        Consider a robot moving in an $m \times n$ grid, where $m\leq n$. It starts at the coordinate $(1,1)$. At each step, when it is at position $(i,j)$, it either moves one square up to $(i+1, j)$ or one square right to $(i, j+1)$. How many ways can the robot travel from $(1,1)$ to $(m,n)$? Justify your answer.

        Answer:
        \begin{align*}
            m - 1 \text{ possible steps up.}               \\
            n - 1 \text{ possible steps right.}            \\
            (m - 1) + (n - 1) \text{ total steps.}         \\
            \text{Each step up and right are the same, so} \\
            \frac{(m+n - 2)!}{(m-1)!(n-1)!}                \\
        \end{align*}
    \end{example}
    \subsection*{Direct proofs}
    \begin{example}
        $6-7\sqrt{2}$ is irrational.
        \begin{align*}
            \text{Assume } 6-7\sqrt{2} = \frac{p}{q} \text{ where } p, q \in \mathbb{Z} \land q \not = 0. \\
            \sqrt{2} = \frac{\left(\frac{p}{q}-6\right)}{-7} = \frac{p-6q}{-7q}                           \\
        \end{align*}
        This is rational as both the numerator and denominator are products and differences of integers and thus are integers, and the denominator is not equal to zero.
    \end{example}
    \subsection*{Pigeonhole Principle}
    \begin{example}
        For every 27 word sequence in a paragraph, at least two words will start with the same letter.

        \textbf{Reason: } There are 27 words that can start with the 26 different English letters. By the pigeonhole principle, two of the words must start with the same letter.
    \end{example}
    \begin{example}
        If a theatre holds 1300 people, how many seats need to be filled to ensure that two people have the same first and last initials?

        \textbf{Answer: } 26 letters in English language, Fname.Lname = 1 letter on each initial. $26 * 26$ means at least 1 person as each combination of initials. $26 * 26 +1$ ensures at least 1 duplicate.
    \end{example}
    \begin{example}
        Let ABC be an equilateral triangle with AB = 1. Show that by selecting 10 points, there are at least two with distance $\leq \frac{1}{3}$ apart.

        \textbf{Answer: } If you split up the triangle by $\frac{1}{3}$ lengths, you will create 9 slots. The 10th will be in the same triangle as another, which is less than $\frac{1}{3}$.
    \end{example}
    \begin{example}
        Prove with the pigeonhole principle that having 100 whole numbers, one can choose 15 of them so that the difference of any 2 is divisible by 7?

        \textbf{Answer: } Use the generalized pigeonhole principle.

        Let $S: \{1\ldots100\}.~S' = S \mod 7 = \{1 \mod 7 \ldots 100 \mod 7\}$

        Partition $S'$ into $7$ subsets of equal size where each subset has same value for $S \mod 7$.

        By generalized PHP, where $\lfloor \frac{units}{holes}\rfloor$ units in each hole, there are at least $\lfloor \frac{100}{7} \rfloor = 14$ remainders in each subset of $S'$.

        $14 * 7 = 98$ which is $2$ less than $100$. We can place those $2$ with the other subsets that contain the same number, increasing the number to at least 15 for one or more of the subsets.

    \end{example}
\end{multicols}
\end{document}
