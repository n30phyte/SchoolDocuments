\documentclass[landscape, letterpaper, 10pt]{article}

\usepackage[margin=.25in]{geometry}
\usepackage{fontspec}
\usepackage{mathtools}
\usepackage{multicol}
\usepackage{unicode-math}

\setmainfont{Calibri}
\setmathfont{Cambria Math}

\begin{document}
\begin{multicols}{3}
    \section*{Financial Statements}
    \begin{tabular}{|l|l|}
        \hline
        Statement Type    & Uses               \\ \hline
        Balance Sheet     & Money at the end   \\ \hline
        Cash Flow         & How much cash made \\ \hline
        Income            & Fiscal profit      \\ \hline
        Retained Earnings & How money is spent \\ \hline
    \end{tabular}

    \subsection*{Balance Sheet}
    \begin{itemize}
        \itemsep0em
        \item Current Assets
              \begin{itemize}
                  \item Cash, Receivables, Inventory, Bills
              \end{itemize}
        \item Long Term Assets
              \begin{itemize}
                  \item Fixed assets: Land, Buildings, Machines, Equipment, Vehicles
                  \item Investments
                  \item Intangibles: Licenses, Patents, Copyrights, Goodwill/PR
              \end{itemize}
        \item Current Liabilities
              \begin{itemize}
                  \item Short term credit, Payables, Expenses, Taxes, Current portion of long term debt/loan
              \end{itemize}
        \item Long Term Liabilities
              \begin{itemize}
                  \item Long term portion of debt, Bonds, Mortgages, Grants
              \end{itemize}
        \item Shareholder's equity
              \begin{itemize}
                  \item Common Stock: Pays dividends depending on company's income
                  \item Preferred Stock: Pays fixed dividends
                  \item Treasury Stock: Bought back stock
                  \item Paid-in capital: More expensive stock
                  \item Retained earnings: Cumulative net income since beginning
              \end{itemize}
    \end{itemize}
    \subsection*{Cash Flow}
    \begin{itemize}
        \item Operating: Production and sale of goods, Depreciation
        \item Investing: New assets, Selling equipment, investments
        \item Financing: Borrowing, Selling stock, Paying debt
    \end{itemize}
    \section*{Ratio Analysis}

    \textbf{Contribution Margin}: the amount of profit each unit makes.

    \textbf{Capital}: obtained from debt and equity.

    \textbf{Debt}: money from bank.

    \textbf{Equity}: money from owners.

    \subsection*{Ratios}
    \begin{align*}
        DebtRatio      & = & \frac{TotalDebt}{TotalAssets}                          \\
        TIE            & = & \frac{TaxableIncome}{Interest}                         \\
        Current        & = & \frac{Assets}{Liabilities}                             \\
        Quick          & = & \frac{AssetsInventory}{Liabilities}                    \\
        IT             & = & \frac{Sales}{AverageInventory}                         \\
        DSO            & = & \frac{Receivables}{AvgSales/day}                       \\
        TAT            & = & \frac{Sales}{TotalAssets}                              \\
        PMoS           & = & \frac{NetIncome}{Revenue}                              \\
        RoA            & = & \frac{NetIncome + i(1-Tax)}{AvgTotalAssets}            \\
        RoE            & = & \frac{NetIncome}{AvgTotalEquity}                       \\
        Price-Earnings & = & \frac{PricePerShare}{EarningsPerShare}                 \\
        BVPS           & = & \frac{TotalEquity - PrefStock}{SharesOutstanding} \\
    \end{align*}
    \section*{Interest Rates}
    \textbf{Effective Interest Rate per Payment Period}
    \begin{align*}
        i & = & \left(1+\frac{r}{M}\right)^C -1 \\
        M & = & CK                              \\
    \end{align*}
    \textbf{Continuous Compounding}
    \begin{align*}
        i_a & = e^r - 1           \\
        i   & = e^\frac{r}{k} - 1 \\
    \end{align*}
    \begin{align*}
        i & : & \text{Effective interest rate per payment period} \\
        M & : & \text{Compounding periods per year}               \\
        C & : & \text{Compounding periods per payment period}     \\
        K & : & \text{Payment periods per year}                   \\
        r & : & \text{Nominal interest rate}                      \\
    \end{align*}
    \section*{Loans}
    \begin{align*}
        A    & : & \text{Annual payments}                       \\
        B_N  & : & \text{Remaining balance at period } N        \\
        B_0  & = & P                                            \\
        B_N  & = & A(P/A, i, N-n)                               \\
        I_N  & : & \text{Interest part of payment at period } N \\
        I_N  & = & B_{N-1} i                                    \\
        PP_N & : & \text{Principal Payment at period } N        \\
        A    & = & PP_N + I_N                                   \\
    \end{align*}
    \section*{Bonds}
    \begin{tabular}{|l|l|}
        \hline
        Market value  & Sum of all present values  \\\hline
        Par value     & Face value                 \\\hline
        Maturity date & When does the par get paid \\\hline
        Coupon rate   & Interest rate              \\\hline
        Cheap Bond    & Discount                   \\\hline
        Expensive     & Premium                    \\\hline
    \end{tabular}
    \begin{align*}
        \text{Yield to maturity} & : & \text{Return if kept to maturity}        \\
                                 & = & A(P/A, i, N) + Par(P/F, i, N)            \\
        \text{Current yield}     & : & \text{Interest payment per market price} \\
                                 & = & \frac{A}{Purchase}                       \\
    \end{align*}
    \section*{Project Analysis}
    \textbf{Never} compare two projects by measuring IRR. Always use MARR and find present worth, or do incremental analysis if no MARR provided.

    \textbf{When project lifespans differ}: If the analysis period is shorter than both projects, us PW analysis. If analysis period is longer, find replacement to pad.

    If analysis period is equal to the highest length, find NPW, do not repeat projects.

    If no analysis period specified, get LCM of projects, and find NPW, repeating to fill out period.

    \section*{Depreciation}
    \subsection*{Book}
    \begin{align*}
        \textbf{Straight Line}                                            \\
        D           & = \frac{P-S}{N}                                     \\
        BV_n        & = P-nD                                              \\
        \textbf {Declining Balance}                                       \\
        d           & = \frac{\text{Multiplier}}{N}                       \\
        D_n         & = dP(1-d)^{n-1}                                     \\
        BV_n        & = P(1-d)^n                                          \\
        \textbf {Sum of Years Digits}                                     \\
        \text{SOYD} & = \frac{N(N+1)}{2}                                  \\
        D_n         & = \frac{N-n+1}{\text{SOYD}}(P-S)                    \\
        \textbf{Units of Production}                                      \\
        D_n         & = \frac{\text{Units/year}}{\text{Total Units}}(P-S) \\
    \end{align*}
    \subsection*{Tax}
    \textbf{Overhaul of equipment} is considered a \textbf{seperate} asset, that gets its own line of depreciation.
    \section*{Taxation}
    \subsection*{Disposal Tax Effects}
    \begin{align*}
        \text{Cost > Salvage > UCC}: G & = t(UCC_N - S) \\
        \text{Cost > UCC > Salvage}: G & = t(UCC_N - S) \\
        \text{Salvage > Cost > UCC}: G & = t(UCC_N - P) \\
                                       & - 0.5t(S-P)    \\
        NS                             & = S + G        \\
    \end{align*}
    \begin{align*}
        NS & : \text{Net Salvage Value}   \\
        G  & : \text{Disposal Tax Effect} \\
        S  & : \text{Salvage Value}       \\
        P  & : \text{Principal}           \\
    \end{align*}
    \subsection*{After Tax Cashflows}

    \textbf{For debt}, interest is taxable, but principal is not. Interest goes to Income Statement, Principal goes to Cash Flow Statement.

    \begin{align*}
        \textbf{Taxable Income} & = \text{Revenue} - \text{Expenses}                   \\
        \textbf{Expenses}       & =  \text{Operating} + \text{CCA} + \text{Interest}   \\
        \textbf{Income Tax}     & = \text{Income} \times \text{Tax Rate}               \\
        \textbf{Net Cash Flow}  & = \text{Net Income} + \text{CCA} +\text{Investments} \\
                                & +\text{Salvage} +\text{Disposal Tax Effect}          \\
    \end{align*}

    \section*{Sensitivity Analysis}
    Seperate the cashflows that you are sure will come, and those that are variable such as revenue, salvage value, fixed costs, etc.
    %WARN: Need more here. Didn't seem too important in class?
    \section*{Replacement Analysis}
    \subsection*{Same Period}
    \textbf{Notation:} $(j_0, n_0), (j_1, n_1), (j_2, n_3), \ldots, (j_i, n_i)$.

    First pair is defender. All following pairs are challengers. Example: $(j_0, 2), (j_1, 5), (j_2, 3)$ means defender stays for 2 years, replaced by challenger 1 for 5 years then challenger 2 for 3 years.

    \textbf{Opportunity Cost Approach} is to find Annual Equivalent Cost of both the defender and challenger, and keep the one with lower costs. Recommended for an infinite planning horizon.

    \textbf{Present Worth Apporach} is to find the present worths of both projects, and choosing one with higher worth. Recommended for a finite planning horizon.

    \textbf{Optimal time to replace} is determined to be when you have the lowest AEC/highest NPW. Trial and error is the only way to find. Easiest method is as follows: Check if immediate replacement fulfils above criteria. If yes, replace immediately. If not, list down all possible AEC/NPWs for both defender and challenger. Find N where AEC/NPW fits criteria.

\end{multicols}
\end{document}