\documentclass[landscape, letterpaper]{article}

\usepackage[margin=.25in]{geometry}
\usepackage{fontspec}
\usepackage{amssymb}
\usepackage{mathtools}
\usepackage{multicol}

\renewcommand{\complement}[1]{{#1}^\mathsf{c}}

\begin{document}
\begin{multicols}{2}
    \section*{Sets}
    \begin{itemize}
        \item Proper Subset: $A \subseteq B$ \textbf{AND} there is one element in B that is \textbf{not} in A
        \item Power Set: $\wp(\{x, y\}) = \{\emptyset, \{x\}, \{y\}, \{x, y\}\}$
    \end{itemize}
    \subsection*{Identities}
    \begin{itemize}
        \item Commutative Law
              \begin{itemize}
                  \item $A \cup B = B \cup A $ and $A \cap B = B \cap A$
              \end{itemize}
        \item Associative Law
              \begin{itemize}
                  \item $(A \cup B) \cup C = A \cup (B \cup C)$ and $(A \cap B) \cap C = A \cap (B \cap C)$
              \end{itemize}
        \item Distributive Law
              \begin{itemize}
                  \item $A \cup (B \cap C) = (A \cup B) \cap (A \cup C)$
                  \item $A \cap (B \cup C) = (A \cap B) \cup (A \cap C)$
              \end{itemize}
        \item Identity Law
              \begin{itemize}
                  \item $A \cup \varnothing = A$ and $A \cap U$
              \end{itemize}
        \item Idempotent Law
              \begin{itemize}
                  \item $A \cup A = A$ and $A\cap A = A$
              \end{itemize}
        \item Universal Bound Law
              \begin{itemize}
                  \item $A \cup U = U$ and $A  \cap \varnothing = \varnothing$
              \end{itemize}
        \item DeMorgan's Law
              \begin{itemize}
                  \item $\complement{(A \cup B)} = \complement{A} \cap \complement{B}$ and $\complement{(A \cap B)} = \complement{A} \cup \complement{B}$
              \end{itemize}
        \item Absorption Law
              \begin{itemize}
                  \item $A \cup (A \cap B) = A$ and $A \cap (A \cup B) = A$
              \end{itemize}
        \item Set Difference Law
              \begin{itemize}
                  \item $A-B = A \cap \complement{B}$
              \end{itemize}
    \end{itemize}
    \section*{Functions}
    A function $f$ from a set $X$ to a set $Y$ denoted by $f: X \to Y$ where $X$ is the domain and $Y$ is the co-domain.
    Every element in $X$ relates to some element in $Y$ and each element in $X$ must map to at most one element in $Y$.
    \subsection*{One-to-One functions}
    A function is called one-to-one or injective when no two elements in $X$ point to the same element in $Y$.
    \subsection*{Onto functions}
    A function is called onto or surjective iff every element in $Y$ has a corresponding $X$.
    \section*{Relations}
    A partition of set $A$ is a collection of nonempty, mutually disjoint subsets whose union is $A$, i.e. $A_1 \cup A_2 \cup A_3 \cup \ldots \cup A_i = A$
    \begin{itemize}
        \item Reflexive $ \iff \forall x \in A, x R x$
              \begin{itemize}
                  \item Each element is related to itself
              \end{itemize}
        \item Symmetric $ \iff \forall x, y \in A, x R y \implies y R x$
              \begin{itemize}
                  \item Each element related to another is related back by that element.
              \end{itemize}
        \item Transitive $\iff \forall x, y, z \in A, (x R y \land y R z)\implies x R z$
              \begin{itemize}
                  \item $A = B = C \therefore A=C$
              \end{itemize}
        \item Partition-induced relation $\iff \forall x, y \in A, x R y \iff (x, y) \in single~partition~A_i$
        \item Equivalence Relation: When a relation is relation is reflexive, symmetric and transitive.
    \end{itemize}


\end{multicols}
\end{document}
