\documentclass[landscape, letterpaper, 8pt]{extarticle}

\usepackage[margin=.25in]{geometry}
\usepackage{fontspec}
\usepackage{amssymb}
\usepackage{mathtools}
\usepackage{multicol}
\usepackage{enumitem}
\usepackage{scrextend}

%\setmainfont{Lato}
%\linespread{0.1}

\renewcommand{\complement}[1]{{#1}^\mathsf{c}}

\begin{document}
\begin{multicols}{2}
    \section*{Logic Properties}
    \begin{itemize}[noitemsep,nolistsep]
        \item \textbf{Generalization: } $p~\therefore p \lor q$ and $q~\therefore p \lor q$
        \item \textbf{Specialization: } $p \land q~\therefore p$ or $q$
        \item \textbf{Conjunction: } $p,~q~\therefore p \land q$
        \item \textbf{Eliminiation: } $p \lor q,~\neg q~\therefore p$
        \item \textbf{Division into Cases: } $p \lor q,~p\implies r,~q\implies r~\therefore r$
        \item \textbf{Contradiction Rule: } $\neg p \implies c~\therefore p$
    \end{itemize}
    \subsection*{Implies}
    \begin{itemize}[noitemsep,nolistsep]
        \item \textbf{Statement: } $p \implies q$
        \item \textbf{Converse: } $q \implies p$
        \item \textbf{Inverse: } $\neg p \implies \neg q$
        \item \textbf{Contrapositive: } $\neg q \implies \neg p$
    \end{itemize}
    \subsection*{Quantified}
    \begin{itemize}[noitemsep,nolistsep]
        \item \textbf{Negation: }$\neg(\forall x, Q(X)) \equiv \exists x, \neg Q(x)$ and $\neg(\exists x, Q(X)) \equiv \forall x, \neg Q(x)$
        \item \textbf{Universal Modus Ponens: } $\forall x, P(x)\implies Q(x),~P(a) \therefore~Q(a)$
        \item \textbf{Universal Modus Tolens: } $\forall x, P(x)\implies Q(x),~\neg Q(a) \therefore \neg P(a)$

    \end{itemize}
    \section*{Induction}
    \subsection*{Strong Induction}
    \textbf{Let} $P(n)$ be the property to be proven, let $a$ and $b$ be fixed integers where $a\leq b$.

    Suppose the following:
    \begin{enumerate}[noitemsep,nolistsep]
        \item \textbf{Basis:} $P(a), P(a+1), \ldots P(b)$ are true.
        \item \textbf{Induction:} For all integers $k\geq b$, if $P(i)$ is true for all integers from $a$ to $k$ then $P(k+1)$ is true.
        \item Hence for all integers $n\geq a, P(n)$ is true.
    \end{enumerate}
    \section*{Number Theory}
    \begin{labeling}{\textbf{Composite}}
        \item [\textbf{Even}] $n$ is even $\iff \exists k, n = 2k$.
        \item [\textbf{Odd}] $n$ is odd $\iff \exists k, n = 2k+1$.
        \item [\textbf{Prime}] $n$ is prime $\iff \forall r, s$, if $n = rs$ then $r=1$ and $s = n$ or vice versa.
        \item [\textbf{Composite}] $n$ is composite $\iff \exists r, s$, if $n = rs$ and $1 < r < s$ and $1 < s < n$.
        \item [\textbf{Rational}] $r$ is rational $\iff \exists a, b$ such that $r = \frac{a}{b}$ and $b \neq 0$.
        \item [\textbf{Divisible}] $d|n \iff \exists k$ such that $n=dk$.
    \end{labeling}
    \section*{Sets}
    Can be proven by using Induction, Algebra or Element method.

    Element Method: Usually with equality or subset relations: prove that an element in LHS is an element in RHS.
    \begin{itemize}[noitemsep,nolistsep]
        \item Proper Subset: $A \subseteq B$ \textbf{AND} there is one element in B that is \textbf{not} in A
        \item Power Set: $\wp(\{x, y, z\}) = \{\emptyset, \{x\}, \{y\}, \{z\},\\
                  \{x, y\},\{x, z\}, \{y, z\}, \{x, y, z\}$
        \item Disjoint set:$A\cap B = \varnothing$
    \end{itemize}
    \subsection*{Identities}
    \begin{itemize}[noitemsep,nolistsep]
        \item Commutative Law
              \begin{itemize}[noitemsep,nolistsep]
                  \item $A \cup B = B \cup A $ and $A \cap B = B \cap A$
              \end{itemize}
        \item Associative Law
              \begin{itemize}[noitemsep,nolistsep]
                  \item $(A \cup B) \cup C = A \cup (B \cup C)$ and $(A \cap B) \cap C = A \cap (B \cap C)$
              \end{itemize}
        \item Distributive Law
              \begin{itemize}[noitemsep,nolistsep]
                  \item $A \cup (B \cap C) = (A \cup B) \cap (A \cup C)$
                  \item $A \cap (B \cup C) = (A \cap B) \cup (A \cap C)$
              \end{itemize}
        \item Identity Law
              \begin{itemize}[noitemsep,nolistsep]
                  \item $A \cup \varnothing = A$ and $A \cap U$
              \end{itemize}
        \item Idempotent Law
              \begin{itemize}[noitemsep,nolistsep]
                  \item $A \cup A = A$ and $A\cap A = A$
              \end{itemize}
        \item Universal Bound Law
              \begin{itemize}[noitemsep,nolistsep]
                  \item $A \cup U = U$ and $A  \cap \varnothing = \varnothing$
              \end{itemize}
        \item DeMorgan's Law
              \begin{itemize}[noitemsep,nolistsep]
                  \item $\complement{(A \cup B)} = \complement{A} \cap \complement{B}$ and $\complement{(A \cap B)} = \complement{A} \cup \complement{B}$
              \end{itemize}
        \item Absorption Law
              \begin{itemize}[noitemsep,nolistsep]
                  \item $A \cup (A \cap B) = A$ and $A \cap (A \cup B) = A$
              \end{itemize}
        \item Set Difference Law
              \begin{itemize}[noitemsep,nolistsep]
                  \item $A-B = A \cap \complement{B}$
              \end{itemize}
        \item Complement Law
              \begin{itemize}[noitemsep,nolistsep]
                  \item $A\cup \complement{A} = U$ and $A \cap \complement{A} = \varnothing$
              \end{itemize}
    \end{itemize}
    \section*{Functions}
    A function $f$ from a set $X$ to a set $Y$ denoted by $f: X \to Y$ where $X$ is the domain and $Y$ is the co-domain.
    Every element in $X$ relates to some element in $Y$ and each element in $X$ must map to at most one element in $Y$.
    \subsection*{One-to-One functions}
    A function is called one-to-one or injective when no two elements in $X$ point to the same element in $Y$.
    \subsection*{Onto functions}
    A function is called onto or surjective iff every element in $Y$ has at least one corresponding element in $X$.
    \subsection*{Inverse functions}
    $F^{-1}(y) = x\text{ where }F(x) = y$.
    $F(x)$ has to be both one-to-one and onto (bijective).
    \section*{Relations}
    A partition of set $A$ is a collection of nonempty, mutually disjoint subsets whose union is $A$, i.e. $A_1 \cup A_2 \cup A_3 \cup \ldots \cup A_i = A$
    \begin{itemize}[noitemsep,nolistsep]
        \item Reflexive $ \iff \forall x \in A, x R x$
              \begin{itemize}[noitemsep,nolistsep]
                  \item Each element is related to itself
              \end{itemize}
        \item Symmetric $ \iff \forall x, y \in A, x R y \implies y R x$
              \begin{itemize}[noitemsep,nolistsep]
                  \item Each element related to another is related back to that element.
              \end{itemize}
        \item Transitive $\iff \forall x, y, z \in A, (x R y \land y R z)\implies x R z$
        \item Partition-induced relation: $\forall x, y \in A, x R y \iff (x, y) \in \text{single partition }A_i$
        \item Equivalence Relation: When a relation is relation is reflexive, symmetric and transitive.
        \item Identity Relation: $I_A = \{(x, y) \in A \times A~|~x=y\}$
    \end{itemize}
    \subsection*{Equivalence Classes}
    The subset of all elements that are related to $a$.

    Suppose $A$ is a set and $R$ is an equivalence relation on $A$. For each element $a$ in $A$, the \textbf{equivalence class of $a$}, $[a]$ or \textbf{class of $a$} is the set of all elements $x$ in $A$ such that $x$ is related to $a$ by $R$. $$[a] = \{x \in A~|~x~R~a\}$$
    \section*{Partial Order Relations}
    A relation that is reflexive, antisymmetric and transitive.
    \subsection*{Antisymmetry}
    A relation is antisymmetric if $x R y \land y R x,~x = y$
    A relation is \textbf{not} antisymmetric if $x R y \land y R x\text{ but }x \neq y$
    \section*{Examples}
    \newtheorem{example}{Example}
    \subsection*{Strong Induction}
    \begin{example}
        \begin{equation*}
            \begin{aligned}
                f(n) = \begin{cases}
                    1                   & \text{if }n=1      \\
                    3                   & \text{if }n=2      \\
                    f(n-1) + 2f(n-2) +3 & \text{if } n\geq 3
                \end{cases}
            \end{aligned}
        \end{equation*}
        Using mathematical induction, prove that for all $n \geq 3: f(n) \geq 2^n$

        Answer:
        \begin{equation*}
            \begin{aligned}
                f(k+1) =f(k)+ 2f(k-1)+3              \\
                \text{Use fact that $f(n) \geq 2^n$} \\
                f(k+1) \geq 2^k+2\cdot 2^{k-1} +3    \\
                \text{and } f(k+1) \geq 2^k + 2^k +3 = 2^{k+1} + 3 \geq 2^{k+1}
            \end{aligned}
        \end{equation*}
    \end{example}
    \begin{example}
        What is the smallest value of k such that any integer postage of k, or
        more, cents can be formed by using only 3-cent and 8-cent stamps?

        Use mathematical induction to prove that all integer postages of k, or more, cents can be
        obtained.
        \begin{equation*}
            \begin{aligned}
                \textbf{Basis: } P(n):                             & n = 3a + 8b       \\
                \textbf{Induction: } P(n+1):                       & n+1 = 3a' + 8b'   \\
                \textbf{Case 1: } \text{if } b \geq 1, n = 3a + 8b & \text{, so}       \\
                n+1 = 3a+1+8b                                      & = 8(b-1) + 9 + 3a \\
                n+1                                                & = 8(b-1) +3(a+3)  \\
                \textbf{Hence }a' = a+3      \text{ and }          & b' = b - 1        \\
                \textbf{Case 2: } \text{if } b = 0, n = 3a         &                   \\
                \text{Since }n \geq 14,~a \geq 5                                       \\
                n+1 = 1 + 15 + 3(l-5)                              & = 16+ 3(l-5)      \\
                \textbf{Hence }a' = l-5      \text{ and }          & b' = 2            \\
            \end{aligned}
        \end{equation*}
    \end{example}

\end{multicols}
\end{document}
