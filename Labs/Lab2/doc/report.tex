\documentclass[12pt]{article}

\usepackage[margin=1.9cm, letterpaper]{geometry}
\usepackage[utf8]{inputenc}
\usepackage{indentfirst}
\usepackage{tikz}
\usepackage{float}
\usepackage{subcaption}
\usepackage{amsmath}
\usepackage{amssymb}
\usepackage{booktabs}
\usepackage{siunitx}
\usepackage{pdfpages}
\usepackage{minted}
\usepackage{fontspec}
\usepackage[style=ieee]{biblatex}

\setmainfont{Noto Serif}
\setsansfont{Segoe UI}
\setmonofont{Consolas}

\begin{document}
\begin{titlepage}
    \begin{center}
        \vspace*{1cm}

        \textbf{LAB 1}

        \vspace{0.5cm}

        Implementing Binary Adders in VHDL

        \vspace{1.5cm}

        \textbf{Michael Kwok (1548454)}

        \vfill

        ECE 410 -- Advanced Digital Logic Design\\
        Department of Electrical and Computer Engineering\\
        University of Alberta\\
        October 6, 2021

    \end{center}
\end{titlepage}

\tableofcontents

\pagebreak

\section{Abstract}

In this report, an exploration of two different approaches to digital logic design was done: register-transfer level and structural.
Two different implementations of a 2 bit full adder was made, and the corresponding testbenches.
The VHDL feature of multiple architecture definitions was used to make the testing and implementation easier,
by only needing to select a specific architecture in Vivado instead of editing files.

\section{Design}

Initially, a
% a. Calculations & Assumptions
% b. Flow Chart 
% c. Truth Table
% d. Synthesized schematic
% e. Board/Implementation (Pin Assignment, On-Board Observations)

\section{Testing \& Simulation}

\section{Conclusion}

\section{References}

\pagebreak
\section{Appendix}

\renewcommand{\thepage}{}

\end{document}
