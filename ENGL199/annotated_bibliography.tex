\documentclass[11pt]{article}
\usepackage[margin=1in, letterpaper]{geometry}
\usepackage[style=ieee, backend=biber]{biblatex}
\usepackage{setspace}

\addbibresource{citations.bib}

\title{Annotated Bibliography of \\ Processor Vulnerabilities due to \\
Speculative Execution}
\author{Michael Kwok (mkwok1@ualberta.ca) \\ University of Alberta \\ Edmonton,
Alberta}
\date{\today}

\doublespacing

\begin{document}
\pagestyle{headings}
\maketitle

\cite{Meltdown} \fullcite{Meltdown}

The authors of the paper are the computer security researchers who discovered
the Meltdown exploit, and in the white paper provide sample outputs that show
the exploit in action. The paper points out one feature of modern CPUs that
allow for the vulnerability to exist, which is called ”Out-of-order execution”.
This feature allows CPUs to cut down on program execution time by running code
that they think will eventually get run in the future and execute them ahead of
time. There is an interesting takeaway from this paper about the disaster, as
this affected one company’s (Intel) CPUs more than their competitors’ (AMD, ARM)
CPUs \cite[13]{Meltdown}, which could point to either an engineering or
management problem inherent in that company. Since the paper is a direct source
from the researchers, the information can be taken as correct and accurate. It
also contains many references to prior research on exploiting this family of
features, but most of the previous research were only theoretical in nature,
and have not been successfully exploited, which makes this discovery novel at
the time. However, the paper does not provide sample code that show a full
implementation of the problem, which makes it difficult for beginning
researchers to fully understand the exploit.

\hfill

\cite{Spectre} \fullcite{Spectre}

Just like the previous paper, this was written by the researchers who
discovered the exploit. In fact, there are some overlapping authors between the
papers. This exploit is much less easily fixed compared to the previous one, and
in this paper, the authors described different ways of "Side channel attacks"
that have been attempted before. This vulnerability affected the same devices from
the same companies as the previous one but is higher in severity as it "will
require fixes to processor designs" \cite[1]{Spectre}. This paper goes into more
detail and explains the background on what was being exploited much more
clearly than the previous paper. Example code is provided for analysis, and
multiple different exploitation techniques were presented. This paper has much
stronger points against the industry's current standard practice of letting their
devices "take shortcuts" and "cut corners" to increase speed and throughput,
disregarding security.

\end{document}