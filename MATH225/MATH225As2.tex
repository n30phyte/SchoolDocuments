\documentclass{article}

\usepackage[letterpaper, margin=1.3cm]{geometry}
\usepackage[utf8]{inputenc}
\usepackage{siunitx}
\usepackage[fleqn]{mathtools}
\usepackage{amsthm}
\usepackage{amssymb}
\usepackage{mathrsfs}
\usepackage{datetime}
\newcommand\aug{\fboxsep=-\fboxrule\!\!\!\fbox{\strut}\!\!\!}

\title{MATH 225 Assignment 2}
\author{Michael Kwok}
\date{2020-07-20}
\begin{document}
\maketitle
\subsection*{1}
A linear transformation must be both additive and homogenous.

Show that the Trace function $Tr : M_{2,2} \rightarrow \mathbb{R}$ is linear, i.e. $Tr(\alpha A + \beta B) = \alpha Tr(A) + \beta Tr(B)$:

let $\alpha, \beta \in \mathbb{R}$

let $A = \begin{bmatrix}
a_1 & b_1\\
c_1 & d_1
\end{bmatrix}$

let $B = \begin{bmatrix}
a_2 & b_2\\
c_2 & d_2
\end{bmatrix}$

\begin{align*}
Tr(\alpha A + \beta B) &= Tr\left(\alpha \begin{bmatrix}
a_1 & b_1\\
c_1 & d_1
\end{bmatrix} + \beta \begin{bmatrix}
a_2 & b_2\\
c_2 & d_2
\end{bmatrix} \right)\\ 
&= Tr\left( \begin{bmatrix}
\alpha a_1 + \beta a_2 & \alpha b_1 + \beta b_2\\
\alpha c_1 + \beta c_2 & \alpha d_1 + \beta d_2
\end{bmatrix} \right)\\
&= \alpha a_1 + \beta a_2 + \alpha d_1 + \beta d_2\\
&= \alpha (a_1 + d_1) + \beta (a_2  + d_2)\\
&= \alpha Tr\left(\begin{bmatrix}
a_1 & b_1\\
c_1 & d_1
\end{bmatrix}\right) + \beta Tr\left(\begin{bmatrix}
a_2 & b_2\\
c_2 & d_2
\end{bmatrix}\right)\\
&= \alpha Tr(A) + \beta Tr(B)
\end{align*}

$\therefore$ the Trace function $Tr : M_{2,2} \rightarrow \mathbb{R}$ is linear
\newpage
\subsection*{2a}
\begin{align*}
    p(x) &= 5(1) + 2(x-1) - (x-1)^2\\
    &= 5 + 2x - 2 - x^2 + 2x - 1\\
    &= 2+ 4x -x^2
\end{align*}
\subsection*{2b}
Define the basis set with standard bases:

\begin{align*}
B_1 = \begin{bmatrix}
1\\
0\\
0\\
\end{bmatrix} B_2 = \begin{bmatrix}
-1\\
1\\
0\\
\end{bmatrix} B_3 = \begin{bmatrix}
1\\
-2\\
1\\
\end{bmatrix}
\end{align*}

Augmented matrix of the system:

\begin{align*}
\begin{bmatrix}
1 & -1 & 1  &\aug & 6 \\
0 & 1  & -2 &\aug & -4\\
0 & 0  & 1  &\aug & 1
\end{bmatrix} \xrightarrow{\text{rref}} \begin{bmatrix}
1 & 0 & 0 &\aug & 3 \\
0 & 1 & 0 &\aug & -2\\
0 & 0 & 1 &\aug & 1
\end{bmatrix}
\end{align*}

\begin{align*}
[q(x)]_{\mathcal{B}} &= \begin{bmatrix}
3\\
-2\\
1\\
\end{bmatrix}
\end{align*}
\subsection*{2c}
Value of $C_\mathcal{B}(a+bx^2+cx^2) = (a+b+c) + (b+2c)(x-1)+c(x-1)^2$

\begin{align*}
    &R((a+b+c)+(b+2c)(x-1)+c(x-1)^2)\\
    &= (a+b+c - b-2c) + (b+2c-c)x + (c-a-b-c) x^2\\
    &= (a-c) + (b+c)x - (a + b)x^2
\end{align*}
\newpage

\subsection*{2d}
For the map $R \circ C_B$ to be invertible, both the change of basis and transformation $R$ must be invertible. A change of basis by definition is invertible, so test for R.

Get the transformation matrix of R by inspection: $
\begin{bmatrix}
1 & -1 & 0 \\
0 & 1 & -1\\
-1 & 0 & 1 
\end{bmatrix}$

Get $det(R)$ by row reduction:

\begin{align*}
det(R) &= det\begin{bmatrix}
1 & -1 & 0 \\
0 & 1 & -1\\
-1 & 0 & 1
\end{bmatrix}\\
&= det\begin{bmatrix}
0 & -1 & 1 \\
0 & 1 & -1\\
-1 & 0 & 1
\end{bmatrix}\\
&= det\begin{bmatrix}
0 & 0 & 0 \\
0 & 1 & -1\\
-1 & 0 & 1
\end{bmatrix}\\
&= det\begin{bmatrix}
1 & 0 & -1 \\
0 & 1 & -1\\
0 & 0 & 0
\end{bmatrix}
\end{align*}

Since matrix is upper triangle, determinant is product of diagonal entries:
\begin{align*}
     det(R) &= 1\cdot1\cdot0\\
     &= 0
\end{align*}

Since the determinant of the transformation matrix is zero, the map is non invertible
\newpage
\subsection*{3a}
Let $p(x) = a + bx + cx^2 + dx^3$.

\begin{align*}
    E_5\left(p\left(x\right)\right) = a+5b+25c+125d
\end{align*}

The image of $E_5 \text{ is all of } \mathbb{R}$. The mapping is onto $\mathbb{R}$ as any value in $\mathbb{R}$ can be represented by replacing $a, b, c,\text{or }d$ with appropriate values in $\mathbb{R}$

Let $k \in \mathbb{R}$.

There must exist some $q(x) \in mathscr{P}_3$ such that $E_5(q(x)) = k$.

\begin{align*}
\text{Let } q(x) &= k + 0x + 0x^2 + 0x^3\\
E_5(q(x)) &= k + 0\cdot5 + 0\cdot25 + 0\cdot125\\
&= k
\end{align*}
$\therefore$ any value $k \in \mathbb{R}$ can be calculated with the appropriate polynomial $q(x) \in \mathscr{P}_3$
\subsection*{3b}
\begin{align*}
    rank(E_5) + nullity(E_5) &= dim(\mathscr{P}_3)\\
    1 + nullity(E_5) &= 4\\
    nullity(E_5) &= 3
\end{align*}
\subsection*{3c}
\begin{align*}
    \text{Let } p(x) &= a+bx+cx^2+dx^3\\
    E_5(p(x)) &= a+5b+25c+125d\\
    ker(E_5) &= \left\{a+bx+cx^2+dx^3: a+5b+25c+125d = 0\right\}
\end{align*}
Use $B$ to denote the basis of $ker(E_5).~B = Nul(ker(E_5))$

Find Null space of the kernel.

Coeffecient matrix of the kernel: $\begin{bmatrix}
1 &5 &25& 125
\end{bmatrix}$

Null space of kernel: $\left\{\begin{bmatrix}
-5\\1 \\0\\ 0
\end{bmatrix}, \begin{bmatrix}
-25\\0 \\1\\ 0
\end{bmatrix}, \begin{bmatrix}
-125\\0 \\0\\ 1
\end{bmatrix}\right\}$

By converting the bases to a form in $\mathbb{R}$ space, we can find the basis of the kernel.
\begin{align*}
    Basis(ker(E_5)) = \left\{-5a+b,-25a+c,-125a+d\right\}
\end{align*}
\end{document}