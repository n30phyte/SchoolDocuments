\documentclass[landscape, letterpaper]{extarticle}

\usepackage[margin=.25in]{geometry}
\usepackage{fontspec}
\usepackage{amssymb}
\usepackage{mathtools}
\usepackage{multicol}
\usepackage{enumitem}
\usepackage{scrextend}
\usepackage{microtype}
\usepackage{bm}
\usepackage[super]{nth}

\renewcommand{\complement}[1]{{#1}^\mathsf{c}}
\DeclarePairedDelimiter{\floor}{\lfloor}{\rfloor}

\begin{document}
\begin{multicols}{3}
    \section*{ER Model}
    \begin{itemize}[noitemsep,nolistsep]
        \item \textbf{Entity:} A distinguishable object, e.g.\ person, thing, object.
        \item \textbf{Relationship:} Represents the fact that certain entities are related to each other.
        \item \textbf{Set:} A set of items with the same type
        \item \textbf{Attribute:} Property of an entity or relationship.
        \item \textbf{Key:} Minimal set of attributes that uniquely identifies entity in set.
        \item \textbf{Role:} Labels added when an entity serves multiple functions in the relationship set.
    \end{itemize}
    \subsection*{Diagram}
    \begin{itemize}[noitemsep,nolistsep]
        \item \textbf{Plain line:} Many to many
        \item \textbf{Bold line:} At least one
        \item \textbf{Arrow:} At most one
        \item \textbf{Bold arrow:} Exactly one
        \item \textbf{Bold arrow one way:} Many to one
              \begin{itemize}[noitemsep,nolistsep]
                  \item \boxed{Employees} -- WorksIn \bm{\rightarrow} \boxed{Departments}
                  \item Each employee can be part of at most one department
              \end{itemize}
        \item \textbf{Bold arrow two way:} One to one
              \begin{itemize}[noitemsep,nolistsep]
                  \item \boxed{Employees} \bm{\leftarrow} Manages \bm{\rightarrow} \boxed{Departments}
                  \item Only one manager per department
              \end{itemize}
    \end{itemize}
    \section*{Relational Algebra}
    \begin{itemize}[noitemsep,nolistsep]
        \item \textbf{Select: } \( \sigma_{condition}(relation) \)
              \begin{itemize}[noitemsep,nolistsep]
                  \item Operators: \( <, \leq, \geq, >, =, \neq \)
                  \item Condition: attribute operator constant OR attribute operator attribute
              \end{itemize}
        \item \textbf{Project: } \( \pi_{attribute~list}(relation) \)
        \item \textbf{Union:} \( A \cup B \)
        \item \textbf{Intersect:} \( A \cap B \)
        \item \textbf{Difference:} \( A - B\)
        \item \textbf{Cartesian Product:} \( A \times B \)
        \item \textbf{Rename:} \( \rho_{(p, q, \ldots, z)}relation\)
        \item \textbf{Join:} \( R \bowtie_{condition} S\)
              \begin{itemize}[noitemsep,nolistsep]
                  \item Equivalent to \(\sigma_{condition} (R \times S)\)
              \end{itemize}
        \item \textbf{Equijoin:} \( R \bowtie_{r = s} S\)
        \item \textbf{Natural join:} \(R \bowtie S \)
              \begin{itemize}[noitemsep,nolistsep]
                  \item Equijoin but with implicit condition
              \end{itemize}
        \item \textbf{Division:} \(R \div S\)
              \begin{itemize}[noitemsep,nolistsep]
                  \item Tuples in R that match all tuples in S
                  \item Derived: \(\pi_n (R) - \pi_n({({\pi_n(R) \times S}) -R})\)
              \end{itemize}
    \end{itemize}
    For set operations (Union, Intersect, Difference), the two relations need to be union compatible i.e.\ same column headers
    \section*{SQL}
    \begin{itemize}[noitemsep,nolistsep]
        \item \verb|CREATE TABLE name(prop type, constraint);|
        \item \verb|DROP TABLE name;|
        \item \verb|ALTER TABLE name action;|
              \begin{itemize}[noitemsep,nolistsep]
                  \item \verb|ADD/MODIFY (prop type/constraint)|
              \end{itemize}
        \item \verb|INSERT INTO name(properties) VALUES(vals);|
        \item \verb|UPDATE name SET prop=val WHERE condition;|
        \item \verb|DELETE FROM name WHERE condition;|
        \item \verb|CREATE VIEW AS SELECT ...;|
    \end{itemize}

    \section*{Disk}
    \textbf{Key:} Improve performance by reducing seek/rotation delays

    \begin{itemize}[noitemsep,nolistsep]
        \item Arrange files sequentially on disk.
              \begin{itemize}[noitemsep,nolistsep]
                  \item Exploit spatial locality.
              \end{itemize}
        \item Read/Write in bigger chunks/pages.
              \begin{itemize}[noitemsep,nolistsep]
                  \item Large page size: related data stored close by might get loaded, avoiding another page transfer
                  \item Small page size: Reduce transfer time, reduce required buffer in main memory
                  \item Typically: 4096 bytes
              \end{itemize}
        \item Buffering reads and writes.
              \begin{itemize}[noitemsep,nolistsep]
                  \item Cache requests so that you don't have to go to disk
                  \item When cache full, throw away least recently used
                  \item Keep track of clean/dirty so if clean no need to write back.
              \end{itemize}
    \end{itemize}
    \section*{Organization}
    \subsection*{Heap File}
    \begin{itemize}[noitemsep,nolistsep]
        \item Rows get appended to end of file as inserted (unordered)
        \item Deleted rows create gaps (Need compaction)
    \end{itemize}
    Performance:
    \begin{itemize}[noitemsep,nolistsep]
        \item Assume File has F pages
        \item Average search: \(F/2\) if exists, \(F\) otherwise
        \item Average delete: \(F/2 + 1\) if exists, \(F\) otherwise
    \end{itemize}
    \subsection*{Sorted File}
    \begin{itemize}[noitemsep,nolistsep]
        \item Rows get inserted based on specific row and everything to shift
              \begin{itemize}[noitemsep,nolistsep]
                  \item Can leave empty space around (fill factor)
                  \item Can use overflow pages (chains)
              \end{itemize}
        \item Deleted rows cause everything to get shifted
    \end{itemize}
    Performance:
    \begin{itemize}[noitemsep,nolistsep]
        \item Assume File has F pages
        \item Average search: \(\log_2 F\) can do binary search
        \item Average insert and delete: \(search + F\) Need to shift rest of file
    \end{itemize}
    \subsection*{Index File}
    Allows to locate rows without having to scan entire table using a search key (not candidate key, only used for search and may not be unique).

    Different structures:

    Integrated Structure:
    \begin{itemize}[noitemsep,nolistsep]
        \item Top contains mechanism for locating index entries.
        \item Bottom contains index entries with full records.
    \end{itemize}

    Secondary Index with Storage structure:
    \begin{itemize}[noitemsep,nolistsep]
        \item Separate file contains mechanism for locating index entries and index entries with pointers to storage.
        \item Storage contains records.
    \end{itemize}

    Index types:
    \begin{itemize}[noitemsep,nolistsep]
        \item Clustered: Data rows are same order as index.
        \item Unclustered: Opposite of above.
        \item Both can be either integrated or split.
        \item Dense: Index for ever record.
        \item Sparse: Index only for some records.
        \item Can be mixed in with the above.
    \end{itemize}

    \section*{Tree Structured Index}
    Can be used to do either an equality search or range search.

    Can be stored in binary tree, every operation will be O(height of tree), or we can use B+ tree instead and store multiple things per node. Complexity will be O(height) but height will be \(\log_{m}(Entries)\) instead of \(\log_{2}(Entries)\)

    \subsection*{ISAM}
    \begin{itemize}[noitemsep,nolistsep]
        \item Static tree, doesn't get updated when insert/delete
        \item Each node is a page
        \item When leaf pages full and you need more space, you add a new overflow page into the chain.
        \item Useful when not many updates
    \end{itemize}

    \subsection*{B+ Tree}
    \begin{itemize}[noitemsep,nolistsep]
        \item Dynamically updates whenever new data is added/removed
        \item Each node still a page.
        \item Search/Insert/Delete guaranteed to be \(\log_{F}(N)\)
        \item Leaf pages form a sequence set (from left to right, they are in order)
        \item Minimum 50\% occupancy, except for the root
        \item When split leaf, key is \textbf{copied} up. When split non-leaf, value is \textbf{moved} up.
        \item On deletion, if below minimum occupancy, rebalance
              \begin{itemize}
                  \item Try to redistribute by taking node with same parent
                  \item If fails, merge
                  \item If merge, delete node
                  \item Propagate
              \end{itemize}
    \end{itemize}

    \subsection*{Hashing}
    \begin{itemize}[noitemsep,nolistsep]
        \item Hash keys into page address
        \item \(H(K): K \rightarrow A\)
        \item Possible hashing functions:
              \begin{itemize}[noitemsep,nolistsep]
                  \item Let last digit of key to be page location
                  \item Folding: Replace text with ASCII code, combine into 4 digits then add, modulus to fit into address space
                  \item Midsquare: Square key and take middle
                  \item Radix transformation: Change radix then modulus
                  \item Same as folding, but with multiplication by index.
              \end{itemize}
        \item \textbf{Key Space:} Set of all possible values for keys
        \item \textbf{Address Space:} Set of all possible values for page address
        \item Address space is usually smaller than key space
        \item Address space must be able to store all records
        \item \textbf{Randomization:} Records should be spread randomly
        \item \textbf{Collision:} Two different keys may have same address
        \item Modulo prime number to increase likelihood of using all spaces
        \item Range and partial key search cannot be done
        \item Equality search is constant time
    \end{itemize}
    \subsection*{Collision}
    Static Methods:
    \begin{itemize}[noitemsep,nolistsep]
        \item Linear Probing
              \begin{itemize}[noitemsep,nolistsep]
                  \item Go down <step> addresses if collision
                  \item Wastes space as rows get deleted
              \end{itemize}
        \item Double Hashing
        \item Separate overflow
    \end{itemize}
    Dynamic Methods:
    \begin{itemize}[noitemsep,nolistsep]
        \item Extendable Hashing
              \begin{itemize}[noitemsep,nolistsep]
                  \item use first k bits of hash, increasing k as collisions happen.
              \end{itemize}
        \item Linear Hashing
    \end{itemize}
    \subsection*{In SQL}
    \begin{itemize}[noitemsep,nolistsep]
        \item Unclustered
              \begin{itemize}[noitemsep,nolistsep]
                  \item \verb|CREATE INDEX idx ON table(prop)|
                  \item Can add \verb|UNIQUE|
              \end{itemize}
        \item Clustered: create table with \verb|WITHOUT ROWID|
    \end{itemize}
    \section*{Normal Forms}
    \begin{itemize}[noitemsep,nolistsep]
        \item Functional Dependency \( X \rightarrow Y\)
              \begin{itemize}[noitemsep,nolistsep]
                  \item \( X \rightarrow YZ \equiv (X \rightarrow Y \text{ and } X \rightarrow Z\))
              \end{itemize}
        \item Armstrong's Axioms
              \begin{itemize}[noitemsep,nolistsep]
                  \item \textbf{Reflexivity:} \(X \rightarrow X \text{ or }XY \rightarrow X\)
                  \item \textbf{Augmentation:} \( X \rightarrow Y: XZ \rightarrow YZ\)
                  \item \textbf{Transitivity:} \( X\rightarrow Y, Y \rightarrow Z: X \rightarrow Z\)
              \end{itemize}
        \item Given a set \(F\) of FDs, the closure of \(F\), \(F^+\) is all the FDs implied from F.
        \item \(X^+\) is the set of all attributes that can be determined by \(X\)
        \item \textbf{Superkey:} Set of attributes that identify row.
        \item \textbf{Candidate Key:} Minimal superkey
        \item \textbf{Key:} Minimal and Unique
        \item \textbf{Unique:} If two tuples have same values, the rest of the attributes must always be the same.
        \item \textbf{Minimal:} Cannot eliminate any attribute without becoming not a key.
        \item \textbf{\nth{3} Normal Form:} All attributes are functionally dependent on solely the primary key.
              \begin{itemize}[noitemsep,nolistsep]
                  \item \(X \rightarrow A\)
                  \item \(X\) is super key of \(R\) or
                  \item \(Y\) is part of a key
              \end{itemize}
        \item \textbf{Boyce-Codd Normal Form:} Stricter 3NF
              \begin{itemize}[noitemsep,nolistsep]
                  \item \(X \rightarrow A\)
                  \item \(X\) is super key of \(R\)
              \end{itemize}
        \item \textbf{Lossless join BCNF:} Not always able to preserve, keep minimal cover
              \begin{itemize}[noitemsep,nolistsep]
                  \item \(X \rightarrow A\) in \(R(Z)\) that violates BCNF
                  \item Split into \(R1\left( X^+ \cap Z \right)\) and
                  \item \(R2\left( \left(Z - X^+\right) \cup X \right)\)
              \end{itemize}
        \item \textbf{Lossless join 3NF:} Start with lossless BCNF decomposition
              \begin{itemize}[noitemsep,nolistsep]
                  \item \(X \rightarrow A\) in \(R(Z)\) that isn't preserved, create relation with \(XA\)
              \end{itemize}
    \end{itemize}
    \section*{Database Tuning}
    \begin{itemize}[noitemsep,nolistsep]
        \item \textbf{Denormalize:} Avoid join but add redundancy
        \item \textbf{Performance Bottleneck:} Heavily used tables, deep index, wide rows
        \item \textbf{Parititioning:} Horizontal: Split rows; Vertical: Split columns + replicate keys
        \item \textbf{Clustered Index:} When point queries with large results, range queries, order-by
        \item \textbf{Unclustered Index:} When point queries with small results, nested loop joins
    \end{itemize}

    \section*{Examples}
    \newtheorem{example}{Example}
    \subsection*{Relational Algebra}
    \begin{example}
        SQL Division
        Define the following tables:

        R = (n, x), S = (x)

        SQL query:
        \begin{verbatim}
CREATE TABLE R(n INTEGER, x TEXT);

CREATE TABLE S(x TEXT);

SELECT n FROM R
EXCEPT
SELECT n FROM
     (SELECT R.n, S.x FROM R, S
     EXCEPT
     SELECT * FROM R);
        \end{verbatim}
    \end{example}
\end{multicols}
\end{document}
